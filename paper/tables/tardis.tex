\begin{deluxetable*}{lcrcccc}
\tabletypesize{\scriptsize}
\tablewidth{0pt}
\tablecaption{\texttt{TARDIS} input parameters\label{tab:tardis}}
\tablehead{
\colhead{Date} &
\colhead{MJD} &
\colhead{Phase} &
\colhead{$t - t_\mathrm{exp}$} &
\colhead{$L$} &
\colhead{$v_\mathrm{boundary}$\tablenotemark{{\scriptsize a}} }&
\colhead{$T_\mathrm{boundary}$\tablenotemark{{\scriptsize b}} } \\
\colhead{(UT) }&
\colhead{} &
\colhead{(d)} &
\colhead{(d)} &
\colhead{($\log L_{\odot}$)} &
\colhead{(\kms) } &
\colhead{(K)}
}
\startdata
2019 Dec 31.277 & 58848.277 & $-14.9$ & 3.0 & 8.55 & 25,000 & 8173 \\
2020 Jan 03.217 & 58851.217 & $-12.0$ & 6.0 & 8.60 & 16,500 & 7925 \\
2020 Jan 15.392 & 58863.392 & $+0.0$ & 18.0 & 9.29 & 10,500 & 9696 \\
\enddata
\tablecomments{Phase is measured in rest-frame days relative to \tbmax. The time of explosion, $t_\mathrm{exp}$, is assumed to be 0.4\,d before \tfl\ for the \texttt{TARDIS} model. }
\tablenotetext{a}{Ejecta velocity at the inner boundary of the photosphere.}
\tablenotetext{b}{Temperature at the inner boundary of the photosphere. 
$T_\mathrm{boundary}$ is not explicitly an input parameter for
\texttt{TARDIS}, it is derived from the luminosity, time since explosion,
inner boundary velocity, and then iteratively updated throughout the simulation.}
\end{deluxetable*}