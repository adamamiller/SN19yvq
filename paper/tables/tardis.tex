\begin{deluxetable*}{lcrrccc}
\tabletypesize{\scriptsize}
\tablewidth{0pt}
\tablecaption{\texttt{TARDIS} input parameters\label{tab:tardis}}
\tablehead{
\colhead{Date} &
\colhead{MJD} &
\colhead{Phase}\tablenotemark{a} &
\colhead{$t - t_\mathrm{exp}$\tablenotemark{b}} &
\colhead{$L$\tablenotemark{c}} &
\colhead{$v_\mathrm{boundary}$\tablenotemark{d} }&
\colhead{$T_\mathrm{boundary}$\tablenotemark{e} } \\
\colhead{(UT) }&
\colhead{} &
\colhead{(d)} &
\colhead{(d)} &
\colhead{($\log L_{\odot}$)} &
\colhead{(\kms) } &
\colhead{(K)}
}
\startdata
2019 Dec 31.277 & 58848.277 & $-14.9$ & 3.0 & 8.55 & 25,000 & 8208 \\
2020 Jan 03.217 & 58851.217 & $-12.0$ & 6.0 & 8.60 & 16,500 & 7353 \\
2020 Jan 15.392 & 58863.392 & $+0.0$ & 18.0 & 9.29 & 10,500 & 7917 \\
\enddata
\tablecomments{\magee{I calculated T myself, but it's possible I didn't
understand your prescription}}
\tablenotetext{a}{Rest-frame time relative to the time of $B$-band maximum,
\tbmax.}
\tablenotetext{b}{Rest-frame time relative to the \texttt{TARDIS} time of
explosion, $t_\mathrm{exp}$.}
\tablenotetext{c}{Emergent Luminosity.}
\tablenotetext{d}{Ejecta velocity at the inner boundary of the photosphere.}
\tablenotetext{e}{Temperature at the inner boundary of the photosphere. The
inner boundary temperature is not explicitly an input parameter for
\texttt{TARDIS}, it is derived from the luminosity, time since explosion, and
inner boundary velocity.}

\end{deluxetable*}
