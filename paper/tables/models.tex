\begin{deluxetable*}{lccccccc}[htp]
\tablecaption{Summary of Observational Properties of \sn \label{tab:models}}
\tablehead{
\colhead{} &
\multicolumn{7}{c}{Does the Model Replicate this Property?} \\
\cline{2-8}
\colhead{Model}
& \colhead{UV flash}
& \colhead{low Luminosity}
& \colhead{fast decline}
& \colhead{red colors}
& \colhead{lack of IGE}
& \colhead{\RSiII\ evolution}
& \colhead{high $v_\mathrm{Si\,II}$}
} 
\startdata
companion interaction & \cmark & \cmark & \xmark & \cmark & \xmark & \xmark & \xmark \\
\radni\ clumps & \textbf{?} (optical flash) & \cmark & \cmark & \cmark & \xmark & \xmark & \cmark \\
He shell double detonation & \cmark & \cmark & \cmark & \cmark & \xmark & \xmark & \xmark \\
violent merger & \textbf{?} & \cmark & \cmark & \cmark & \xmark & \cmark & \xmark
\enddata
%
\tablecomments{If a model replicates a specific property we show a \cmark,
whereas properties that are not matched are signified with a \xmark. Ambiguous
cases are shown as \textbf{?}. The companion-interaction model presented in
\citet{Kasen10a} does not detail the properties of the SN after the ejecta
collide with the companion. For the maximum-light properties, we compare \sn\
to the lowest-luminosity Chandrasekhar-mass explosion presented in
\citet[][their N1600C model]{Sim13}. The violent merger model does not track
circumstellar material, it is unclear if interaction between the ejecta and
unbound material could reproduce the UV flash (see text). While all the models
show evidence for IGE absorption in the early spectra, the double detonation
and \radni\ clump models in particular show strong IGE absorption and line
blanketing that is not observed in \sn.}
%
\end{deluxetable*}