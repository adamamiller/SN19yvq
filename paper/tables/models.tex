\begin{deluxetable*}{lccccccc}[htp]
\tabletypesize{\small}
\tablecaption{Summary of Observational Properties of \sn \label{tab:models}}
\tablehead{
\colhead{} &
\multicolumn{7}{c}{Does the Model Replicate this Property?} \\
\cline{2-8}
\colhead{}
& \colhead{UV}
& \colhead{Low Peak}
& \colhead{Intermediate/Fast}
& \colhead{Red Colors}
& \colhead{Lack of IGE}
& \colhead{\RSiII}
& \colhead{High \ion{Si}{II}} \\
\colhead{Model}
& \colhead{Flash}
& \colhead{Luminosity}
& \colhead{Decline}
& \colhead{at All Epochs}
& \colhead{in Early Spectra}
& \colhead{Evolution}
& \colhead{Velocities}
} 
\startdata
Companion interaction & \cmark & \textbf{?} & \textbf{?} & \textbf{?} & \textbf{?} & \textbf{?} & \textbf{?} \\
\radni\ clumps & \cmark & \textbf{?} & \textbf{?} & \cmark & \xmark & \textbf{?} & \textbf{?} \\
He shell double detonation & \cmark & \cmark & \cmark & \cmark & \xmark & \xmark & \xmark \\
Violent merger & \textbf{?} & \cmark & \cmark & \cmark & \cmark & \cmark & \xmark
\enddata
%
\tablecomments{If a model replicates a specific property we show a \cmark,
whereas properties that are not matched are signified with an \xmark. Ambiguous
cases are shown as \textbf{?}. An important distinction for the
companion-interaction and \radni-clump models is that they are empirical,
whereas the double-detonation and violent merger models are based on a specific
realization of an exploding WD. Given that the companion-interaction and
\radni-clump models do not model the explosion itself, we label all properties
that are not generic to the class as \textbf{?}. While the double-detonation
and \radni-clump models produce a UV flash, it is unclear whether or not they
can match the magnitude of the observed flash in \sn. The violent merger model
does not track circumstellar material, and additional simulations are needed to
understand whether interaction between the ejecta and unbound material could
reproduce the UV flash (see text). In addition to showing evidence for IGE
absorption in the early spectra, the double-detonation and \radni-clump models
show strong IGE absorption and line blanketing around maximum light that is not
observed in \sn.}
%
\end{deluxetable*}