%% using aastex version 6.3
\documentclass[twocolumn]{aastex63}

\newcommand{\vdag}{(v)^\dagger}
\newcommand\aastex{AAS\TeX}
\newcommand\latex{La\TeX}

%%%%%%%%%%%%%%%%%%%%%%%%%%%%%%%%%%%%%%%%%%%%%%%%%%%%%%%%%%%%%%%%%%%%%%%%%%%%%%%%
%%
%% The following section defines new commands for comments from co-authors
%%
\definecolor{DarkOrange}{RGB}{204, 85, 0}
\definecolor{LincolnGreen}{RGB}{17, 102, 0}
\definecolor{Rust}{HTML}{9B4F0F}
\definecolor{DarkCyan}{HTML}{008B8B}
\definecolor{MediumAquaMarine}{HTML}{66CDAA}


\def\ion#1#2{#1$\;${\footnotesize\rm{#2}}\relax}

\newcommand{\kate}[1]{{\color{red} KM: \textbf{#1}}}
\newcommand{\steve}[1]{{\color{DarkCyan} Steve S: \textbf{#1}}}
\newcommand{\magee}[1]{{\color{Rust} MM: \textbf{#1}}}
\newcommand{\abi}[1]{{\color{LincolnGreen} AP: \textbf{#1}}}
\newcommand{\yy}[1]{{\color{blue} YY: \textbf{#1}}}
\newcommand{\aam}[1]{{\color{DarkOrange} aam: \textbf{#1}}}

\newcommand{\fromkate}[1]{{\color{brown} fromKM: {#1}}}
\newcommand{\frommark}[1]{{\color{orange} fromMM: {#1}}}
\newcommand{\frommb}[1]{{\color{purple} fromMB: {#1}}}
\newcommand{\fromabi}[1]{{\color{teal} fromAP: {#1}}}

\newcommand{\stockholm}[1]{{\color{cyan} stockholm: {#1}}}
\newcommand{\todo}[1]{{\color{magenta} to-do: {#1}}}

\newcommand{\rztf}{$r_\mathrm{ZTF}$}
\newcommand{\gztf}{$g_\mathrm{ZTF}$}
\newcommand{\iztf}{$i_\mathrm{ZTF}$}
\newcommand{\tfl}{$t_\mathrm{fl}$}
\newcommand{\trise}{$t_\mathrm{rise}$}
\newcommand{\tbmax}{$T_{B,\mathrm{max}}$}
\newcommand{\kms}{km\,s$^{-1}$}
\newcommand{\RSiII}{$\mathcal{R}($\ion{Si}{II}$)$}
\newcommand{\radni}{$^{56}$Ni}

\newcommand{\sn}{SN\,2019yvq}

\usepackage{lineno}
\linenumbers


%%
%%%%%%%%%%%%%%%%%%%%%%%%%%%%%%%%%%%%%%%%%%%%%%%%%%%%%%%%%%%%%%%%%%%%%%%%%%%%%%%%

%% Reintroduced the \received and \accepted commands from AASTeX v5.2
\received{\today}
\revised{}
\accepted{}
%% Command to document which AAS Journal the manuscript was submitted to.
%% Adds "Submitted to " the argument.
\submitjournal{ApJ}

%% For manuscript that include authors in collaborations, AASTeX v6.3
%% builds on the \collaboration command to allow greater freedom to 
%% keep the traditional author+affiliation information but only show
%% subsets. The \collaboration command now must appear AFTER the group
%% of authors in the collaboration and it takes TWO arguments. The last
%% is still the collaboration identifier. The text given in this
%% argument is what will be shown in the manuscript. The first argument
%% is the number of author above the \collaboration command to show with
%% the collaboration text. If there are authors that are not part of any
%% collaboration the \nocollaboration command is used. This command takes
%% one argument which is also the number of authors above to show. A
%% dashed line is shown to indicate no collaboration. This example manuscript
%% shows how these commands work to display specific set of authors 
%% on the front page.
%%
%% For manuscript without any need to use \collaboration the 
%% \AuthorCollaborationLimit command from v6.2 can still be used to 
%% show a subset of authors.
%
%\AuthorCollaborationLimit=2
%
%% will only show Schwarz & Muench on the front page of the manuscript
%% (assuming the \collaboration and \nocollaboration commands are
%% commented out).
%%
%% Note that all of the author will be shown in the published article.
%% This feature is meant to be used prior to acceptance to make the
%% front end of a long author article more manageable. Please do not use
%% this functionality for manuscripts with less than 20 authors. Conversely,
%% please do use this when the number of authors exceeds 40.
%%
%% Use \allauthors at the manuscript end to show the full author list.
%% This command should only be used with \AuthorCollaborationLimit is used.

%% The following command can be used to set the latex table counters.  It
%% is needed in this document because it uses a mix of latex tabular and
%% AASTeX deluxetables.  In general it should not be needed.
%\setcounter{table}{1}

%%%%%%%%%%%%%%%%%%%%%%%%%%%%%%%%%%%%%%%%%%%%%%%%%%%%%%%%%%%%%%%%%%%%%%%%%%%%%%%%
%%
%% The following section outlines numerous optional output that
%% can be displayed in the front matter or as running meta-data.
%%
%% If you wish, you may supply running head information, although
%% this information may be modified by the editorial offices.
\shorttitle{\sn\ is Fun and Cool}
\shortauthors{Miller et al.}
%%
%% You can add a light gray and diagonal water-mark to the first page 
%% with this command:
\watermark{DRAFT}
%% where "text", e.g. DRAFT, is the text to appear.  If the text is 
%% long you can control the water-mark size with:
%% \setwatermarkfontsize{dimension}
%% where dimension is any recognized LaTeX dimension, e.g. pt, in, etc.
%%
%%%%%%%%%%%%%%%%%%%%%%%%%%%%%%%%%%%%%%%%%%%%%%%%%%%%%%%%%%%%%%%%%%%%%%%%%%%%%%%%
\graphicspath{{./}{figures/}}
%% This is the end of the preamble.  Indicate the beginning of the
%% manuscript itself with \begin{document}.

\begin{document}

\title{The Spectacular Ultraviolet Flash From the \\ Type Ia Supernova 2019yvq}

%% LaTeX will automatically break titles if they run longer than
%% one line. However, you may use \\ to force a line break if
%% you desire. In v6.3 you can include a footnote in the title.

%% A significant change from earlier AASTEX versions is in the structure for 
%% calling author and affiliations. The change was necessary to implement 
%% auto-indexing of affiliations which prior was a manual process that could 
%% easily be tedious in large author manuscripts.
%%
%% The \author command is the same as before except it now takes an optional
%% argument which is the 16 digit ORCID. The syntax is:
%% \author[xxxx-xxxx-xxxx-xxxx]{Author Name}
%%
%% This will hyperlink the author name to the author's ORCID page. Note that
%% during compilation, LaTeX will do some limited checking of the format of
%% the ID to make sure it is valid. If the "orcid-ID.png" image file is 
%% present or in the LaTeX pathway, the OrcID icon will appear next to
%% the authors name.
%%
%% Use \affiliation for affiliation information. The old \affil is now aliased
%% to \affiliation. AASTeX v6.3 will automatically index these in the header.
%% When a duplicate is found its index will be the same as its previous entry.
%%
%% Note that \altaffilmark and \altaffiltext have been removed and thus 
%% can not be used to document secondary affiliations. If they are used latex
%% will issue a specific error message and quit. Please use multiple 
%% \affiliation calls for to document more than one affiliation.
%%
%% The new \altaffiliation can be used to indicate some secondary information
%% such as fellowships. This command produces a non-numeric footnote that is
%% set away from the numeric \affiliation footnotes.  NOTE that if an
%% \altaffiliation command is used it must come BEFORE the \affiliation call,
%% right after the \author command, in order to place the footnotes in
%% the proper location.
%%
%% Use \email to set provide email addresses. Each \email will appear on its
%% own line so you can put multiple email address in one \email call. A new
%% \correspondingauthor command is available in V6.3 to identify the
%% corresponding author of the manuscript. It is the author's responsibility
%% to make sure this name is also in the author list.
%%
%% While authors can be grouped inside the same \author and \affiliation
%% commands it is better to have a single author for each. This allows for
%% one to exploit all the new benefits and should make book-keeping easier.
%%
%% If done correctly the peer review system will be able to
%% automatically put the author and affiliation information from the manuscript
%% and save the corresponding author the trouble of entering it by hand.

% \author[0000-0001-9515-478X]{A.~A.~Miller}
% \affiliation{Center for Interdisciplinary Exploration and Research in Astrophysics (CIERA) and Department of Physics and Astronomy, Northwestern University, 1800 Sherman Road, Evanston, IL 60201, USA}
% \affiliation{The Adler Planetarium, Chicago, IL 60605, USA}
% \email{amiller@northwestern.edu}

\author{ZTF}

\author{et al.}

%% Note that the \and command from previous versions of AASTeX is now
%% depreciated in this version as it is no longer necessary. AASTeX 
%% automatically takes care of all commas and "and"s between authors names.

%% Mark off the abstract in the ``abstract'' environment. 
\begin{abstract}

Early observations of Type Ia supernovae (SNe\,Ia) provide essential clues for
understanding the progenitor system that gave rise to the terminal
thermonuclear explosion. We present exquisite observations of \sn, the second
ever SN\,Ia, after iPTF\,14atg, to display an early flash of emission in the
ultraviolet (UV) and optical. Our analysis finds that \sn\ was unusual, even
when ignoring the initial flash, in that it was moderately underluminous for a
SN\,Ia ($M_g \approx -18.5$\,mag at peak) yet featured very high absorption
velocities ($v \approx 15,000$\kms\ for \ion{Si}{II} $\lambda$6355 at peak).
We find that many of the observational features of \sn, aside from the flash,
can be explained if the explosive yield of radioactive \radni\ is relatively
low and is highly stratified and constrained to the innermost layers of the
ejecta. To explain both the UV/optical flash and peak properties of \sn\ we
consider four different models: interaction between the SN ejecta and a
non-degenerate companion, extended clumps of \radni\ in the outer ejecta, a
double detonation explosion, and the violent merger of two white dwarfs. Each
of these models has shortcomings when compared to the observations, though we
favor either a double detonation or white dwarf merger as the most likely
explanation for \sn, as they are more likely to match the observations with
minor tuning. In closing, we predict that the nebular spectra of \sn\ will
feature strong [\ion{Ca}{II}] emission, if it was a double detonation, or
narrow [\ion{O}{I}] emission, if it was due to a violent merger.

\end{abstract}

%% Keywords should appear after the \end{abstract} command. 
%% See the online documentation for the full list of available subject
%% keywords and the rules for their use.
\keywords{}

\section{Introduction} \label{sec:intro}

\todo{it is possible/likely I did not cite your favorite paper below, feel
free to provide suggestions}

There is now no doubt that Type Ia supernovae (SNe\,Ia) are the result of
thermonuclear explosions in C/O white dwarfs (WDs) in multiple star systems
\citep[see e.g.,][and references therein]{Maoz14}. Despite this certainty, the
nature of the binary companion, which plays an essential role in driving the
primary WD towards explosion, remains highly uncertain.

Historically, most studies have focused on whether or not the companion is
also a WD, the double degenerate (DD) scenario \citep[e.g.,][]{Webbink84}, or
some other non-degenerate star, the single degenerate (SD) scenario
\citep[e.g.,][]{Whelan73}. In addition to this fundamental question, recent
efforts have also focused on whether or not sub-Chandrasekhar mass WDs can
explode \citep[e.g.,][]{Fink10,Shen14,Scalzo14a,Polin19} and the specific
scenario in which the WD explodes \citep[see][and references
therein]{Hillebrandt13,Ropke18}.

Unfortunately, maximum light observations of SNe\,Ia have not provided the
discriminatory power necessary to answer these questions and infer the
progenitor system.\footnote{Indeed, SNe Ia are standardizable candles
precisely because they are so uniform at this phase.} It has recently been
recognized that extremely early observations, in the hours to days after
explosion, may help to constrain which progenitor scenarios are viable and
which are not. In particular, \citet{Kasen10a} showed that for favorable
configurations in the SD scenario, the SN ejecta will collide with the
non-degenerate companion producing a shock that gives rise to an
ultraviolet/optical flash in excess of the typical emission from a SN\,Ia.

The findings in \citet{Kasen10a} launched a bevy of studies to search for
such a signal \citep[e.g.,][]{Hayden10,Ganeshalingam11,Bianco11,Olling15},
including several claims of a detection of the interaction with a
non-degenerate companion (e.g.,
\citealt{Cao15,Marion16,Hosseinzadeh17,Dimitriadis19}; though see also
\citealt{Kromer16,Shappee16a,Shappee19} for alternative explanations). In the
meantime, it has been found that an early optical bump, or flash, in the light
curves of SNe\,Ia is not uniquely limited to the SD scenario
\citep[e.g.,][]{Raskin13,Piro16,Levanon17,Noebauer17,Polin19,Magee20a}.

Despite some observational degeneracies, early observations have and will
continue to play a critical role in understanding the progenitors of SNe\,Ia
\citep[e.g., early photometry of SN\,2011fe constrained the size of the
exploding star to be $\lesssim 0.02$\,$R_\odot$, providing the most direct
evidence to date that SNe\,Ia come from WDs;][]{Bloom12a}.

Here we present X-ray, ultraviolet (UV), and optical observations of the
spectacular \sn, only the second ever SN\,Ia, after iPTF\,14atg \citep{Cao15},
to exhibit an early UV flash.\footnote{``Excess'' emission or early optical
bumps have been observed and claimed in many other SNe\,Ia
\citep[e.g.,][]{Marion16,Hosseinzadeh17,Shappee19,Dimitriadis19}. These events
lack a distinct early decline in the UV, however, which distinguishes
iPTF\,14atg and \sn.} \sn\ declined by $\sim$2.5\,mag in the UV in the
$\sim$3\,d after discovery followed by a more gradual rise and fall, typical
of SNe\,Ia, in the ensuing weeks. Our observations and analysis show that,
even if the early flash had been observationally missed, we would conclude
that \sn\ is unusual relative to normal SNe\,Ia. We consider several distinct
models to explain the origin of \sn\ and find that they all have considerable
shortcomings. Spectroscopic observations of \sn\ obtained during the nebular
phase will narrow the range of potential explanations for this highly unusual
explosion.

Alongside this paper, we have released our open-source analysis and the data
utilized in this study. These are available online at
\href{https://github.com/adamamiller/SN19yvq}{\url{https://github.com/adamamiller/SN19yvq}}.

\section{Discovery and Observations}\label{sec:obs}

\sn\ was discovered by K.~Itagaki, and detected at an unfiltered magnitude
of 16.7\,mag, in an image obtained on 2019 Dec 28.74 UT.\footnote{UT times
are used throughout this paper.} The transient candidate was announced
$\sim$2\,hr later on the Transient Name Server (TNS), and given the
designation AT\,2019yvq \citep{Itagaki19}. Subsequent spectroscopic
observations confirmed the SN nature of the transient, with an initial
report that the event was a SN\,Ib/c, and subsequent spectra confirming the
event as a SN\,Ia.\footnote{The initial classification is from
\citet{Kawabata20}, while the SN\,Ia classifications are from Prentice,
Mazzali, Teffs \& Medler and Dahiwale \& Fremling (see
\url{https://wis-tns.weizmann.ac.il/search?&name=SN2019yvq}).} These
spectroscopic observations also showed \sn\ to be at the same redshift as
NGC\,4441, its host galaxy.

\subsection{ZTF Photometric Observations}

The Zwicky Transient Facility (ZTF; \citealt{Bellm19,Graham19,Dekany20})
simultaneously conducts multiple time-domain surveys using the ZTF camera on
the the Palomar Oschin Schmidt 48 inch (P48) telescope. \sn\ was first
detected by ZTF on 2019 Dec 29.46, as part of the ZTF public survey (see
\citealt{Bellm19a}). The automated ZTF pipeline \citep{Masci19} automatically
detected \sn, which passed internal thresholds (e.g., \citealt{Mahabal19}),
leading to the production and dissemination of a real-time alert
\citep{Patterson19} and the internal designation ZTF\,19adcecwu. The public
alert included the position, $\alpha = 12^{\mathrm{h}}27\arcmin21\farcs836$,
$\delta = +64\degr47\arcmin59\farcs87$ (J2000), and brightness, \rztf$ =
17.14\pm0.05$\,mag, which, together with the \citet{Itagaki19} discovery
report suggested the SN was fading. Continued monitoring with ZTF, and
follow-up with other telescopes, confirmed a spectacular decline in the early
emission from \sn\ (Figure~\ref{fig:p48}).

\begin{figure*}
    \centering
    \includegraphics[width=6in]{./figures/P48_lc.pdf}
    %
    \caption{Photometric evolution of \sn, highlighting the initial decline
    observed in the light curve. \gztf, \rztf, and \iztf\ observations are
    shown as filled green circles, open red circles, and filled golden
    crosses, respectively. UVOT $uvw1$ and $uvm2$ are shown as filled and
    open squares, respectively. The lower axis shows time measured in
    rest-frame days relative to the time of first light, \tfl\ (see
    \S\ref{sec:phot}), while the upper axis shows time relative to the time
    of $B$-band maximum, \tbmax. Note that the horizontal axis is shown with
    a linear scale from $0\,\mathrm{d} \le t - t_\mathrm{fl} \le 3$\,d and a
    log scale for $t - t_\mathrm{fl} > 3$\,d. Vertical grey ticks show
    epochs of spectroscopic observations.}
    %
    \label{fig:p48}
\end{figure*}

The field of \sn\ was additionally observed by ZTF with nearly a nightly
cadence as part of the ZTF partnership Uniform Depth Survey (ZUDS;
D.~Goldstein et al., in prep.). Using images obtained as part of the ZUDS
program, we perform forced PSF photometry at the location of \sn\ following
the procedure described in \citet{Yao19}.\footnote{Images obtained as part of
the ZTF public survey have not been released, preventing us from applying our
forced-PSF measurements. We therefore only include forced-PSF measurements in
the analysis described herein, though we note that our measurements are
largely consistent with those provided in the public alerts.} The evolution
of \sn\ in the \gztf, \rztf, and \iztf\ filters is shown in
Figure~\ref{fig:p48}.

\subsection{\textit{Swift} UVOT and XRT Observations}\label{sec:swift}

Ultraviolet (UV) observations of \sn\ were obtained with the
Ultra-Violet/Optical Telescope (UVOT; \citealt{Roming05}) onboard the Neil
Gehrels Swift Observatory (hereafter \textit{Swift}; \citealt{Gehrels04})
following a time-of-opportunity request.\footnote{\textit{Swift} ToO requests
for \sn\ (\textit{Swift} Target ID: 13037) have been submitted by
D.~Hiramatsu, J.~Burke, and S.~Schulze.} Pre-SN UVOT reference images are
limited to the $uvw1$, $uvm2$, and $uvw2$ filters. As a result we cannot
provide accurate estimates of the SN flux in the \textit{Swift} $u$, $b$, and
$v$ filters. We estimate the flux in the $uvw1$, $uvm2$, and $uvw2$ filters
using a circular aperture with a $3\arcsec$ radius at the SN position, and
subtract the flux measured using an identical procedure in the pre-SN\,Images.
For clarity, we only show the \textit{Swift} $uvw1$ and $uvm2$ light curves in
Figure~\ref{fig:p48}.\footnote{The $uvw2$ evolution is nearly identical to
$uvm2$. Furthermore, the red leak associated with the $uvw2$ filter (see e.g.,
\citealt{Breeveld11}), in combination with the relatively red spectral energy
distribution of SNe\,Ia, make it very difficult to interpret $uvw2$ light
curves of SNe\,Ia (see \citealt{Brown17} and references therein). Therefore,
unless otherwise noted, we exclude $uvw2$ measurements from the analysis
below.} \textit{Swift}/UVOT observations show that the initial decline seen in
the optical is even more dramatic in the UV.

While absolute flux measurements in the UVOT $u$, $b$, and $v$ filters are
not available, assuming the underlying flux from the host is constant, we
can estimate the time of $B$-band maximum, \tbmax, from the relative
$b$-band light curve. Using a second-order polynomial, we model the $b$-band
light curve near peak (including observations between JD$\,> \,$2,458,855.5
and JD$\,<\,$2,458,871.5). From this fit we estimate \tbmax$ =
$2,458,863.83$ \,\pm \,0.21$\,JD.

In parallel with the \textit{Swift}/UVOT observations, \textit{Swift} observed
\sn\ with its onboard X-ray telescope \citep[XRT;][]{Burrows05} between 0.3
and 10\,keV in the photon counting mode. We analyzed the data with the
online-service of the UK \textit{Swift}
team\footnote{\href{https://www.swift.ac.uk/user_objects/}{\url{https://www.swift.ac.uk/user_objects/}}} that uses the methods described in \citet{Evans07}
and \citet{Evans09} and the software package \texttt{HEASOFT}\footnote{
\href{https://heasarc.gsfc.nasa.gov/docs/software/heasoft/}{\url{https://heasarc.gsfc.nasa.gov/docs/software/heasoft/}}} version 6.26.1 \citep{Heasarc}.

\begin{deluxetable}{rDD}
\tabletypesize{\scriptsize}
\tablewidth{0pt}
\tablecaption{\textit{Swift}/XRT photometry\label{tab:xrt}}
\tablehead{
\colhead{Phase} &
\multicolumn2c{Count Rate} &
\multicolumn2c{Flux} \\
\colhead{(ks)} &
\multicolumn2c{($10^{-3}$\,ct\,s$^{-1}$)} &
\multicolumn2c{$(10^{-14}\,\mathrm{erg\,cm}^{-2}\,\mathrm{s}^{-1})$}
}
\decimals
\startdata
250  &    1.6  $^{+0.6}_{-0.5 }$ &     5.4 $^{+2.1}_{-1.7}$ \\
750  &    1.3  $^{+0.8}_{-0.6 }$ &     4.5 $^{+2.7}_{-2.0}$ \\
1250 &    2.0  $^{+1.1}_{-0.8 }$ &     6.9 $^{+3.9}_{-2.9}$ \\
1750 &    3.3  $^{+1.9}_{-1.4 }$ &    11.4$^{+6.6}_{-4.9}$ \\
2250 & $<$5.4                    & $<$18.8         \\
3750 &$<$17.9                    & $<$62.7         \\
4250 &    2.6  $^{+1.7}_{-1.2 }$ &     8.9$^{+5.8}_{-4.1}$ \\
4750 &    2.2  $^{+1.5}_{-1.1 }$ &     7.8$^{+5.3}_{-3.8}$ \\
5250 & $<$8.4                    & $<$29.3         \\
\enddata
\tablecomments{The count rate and flux are reported in the passband from 0.3 to 10~keV. The flux is corrected for absorption. The reference time is MJD=58846.895949 (29 December 2019 at 21:30:15 UT).}
\end{deluxetable}

To build the light curve, we divided the time-interval of the \textit{Swift}
campaign into blocks of 500\,ks (i.e., 5.8\,d). Table~\ref{tab:xrt} summarizes
our measurements within these intervals, where the photon count rate varies
between 0.001 and 0.003\,ct\,s$^{-1}$, with a signal-to-noise ratio between
$\sim$2.6 and 3.4. In 3 of these binned epochs we do not detect any
significant X-ray emission, and therefore report $3\sigma$ upper limits. To
convert the count-rate to flux, we build a spectrum from the entire data set
and model the data as an absorbed power-law. In this model, the absorption
components reflect the Galactic absorption towards \sn\ ($N_{\rm
H,\,X}=2.28\times10^{20}\,{\rm cm}^{-2}$)) (unclear where this value comes
from. doesn't match nH column density calculator) and the absorption in the
host galaxy. The photon index $\Gamma$ of the power law ($f \propto
E^{-\Gamma}$) is $2.0^{+0.9}_{-0.5}$ and the host absorption is
$7^{+207}_{-7}\times10^{20}\,{\rm cm}^{-2}$ (90\% confidence). The goodness of
fit is $\chi^2=30.33$ for 31 degrees of freedom
(w-statistics\footnote{\href{http://heasarc.gsfc.nasa.gov/xanadu/xspec/manual/XSappendixStatistics.html}{\url{http://heasarc.gsfc.nasa.gov/xanadu/xspec/manual/XSappendixStatistics.html}}}). The unabsorbed energy conversion factor is
$3.5\times10^{-11}~\mathrm{erg\,cm}^{-2}\,\mathrm{ct}^{-1}$.

Given the constant X-ray emission (to within the uncertainties, see
Table~\ref{tab:xrt}), we conclude that this emission is from a background
source, and not \sn. A lack of X-ray emission is consistent with SNe\,Ia (e.g.,
\citealt{Margutti12}). \todo{Note - if CSM interaction becomes the story of
this SN, then this paragraph needs to be revisited.}

\subsection{Optical Spectroscopy}

Spectroscopic observations of \sn\ were taken with a variety of telescopes
and instruments over multiple epochs beginning $\sim$2\,d after discovery
and continuing through $\sim$2\,months after \tbmax. An observing log is
listed in Table~\ref{tab:spectra}. The spectra were reduced using standard
procedures in \texttt{IDL}/\texttt{Python}/\texttt{Matlab}. The optical
spectral evolution of \sn\ is illustrated in Figure~\ref{fig:spec_evo}.

\begin{figure}
    \centering
    \includegraphics[width=3.35in]{./figures/spec_evo.pdf}
    %
    \caption{Observed spectral sequence of \sn. Spectra have been normalized
    by their median flux between 7200\,\AA\ and 7400\,\AA. The phase of each
    observation relative to \tbmax\ is shown to the right of the individual
    spectra. Prominent spectral features from intermediate mass elements are
    highlighted with vertical dashed lines. Some of the spectra show
    imperfect Telluric subtractions, giving rise to the non-smooth features
    around $\lambda_\mathrm{obs} \approx 7600$\,\AA.}
    %
    \label{fig:spec_evo}
\end{figure}

\section{NGC\,4441: the Host of \sn}\label{sec:host}

NGC\,4441 is the host galaxy of \sn. Sloan Digital Sky Survey (SDSS;
\citealt{York00}) spectroscopic measurements of the nucleus of NGC\,4441
yield a heliocentric-recession velocity of 2663\,\kms\ ($z_\mathrm{helio} =
0.00888$; \citealt{Abolfathi18}) and a \texttt{STARBURST} classification for
NGC\,4441. Morphologically, NGC\,4441 is classified as a peculiar,
weakly-barred, late-type lenticular galaxy (SABO$+$ pec;
\citealt{de-Vaucouleurs91}). SDSS images show a clear dust lane near the
center of the galaxy.

Using the 2M++ model of \citet{Carrick15}, we estimate a peculiar velocity
towards NGC\,4441 of $+328.6$\,\kms, which combined with the recession
velocity in the frame of the cosmic microwave background\footnote{See
\url{https://ned.ipac.caltech.edu/velocity_calculator}} (CMB, $v_\mathrm{CMB}
= 2770.6$\,\kms), yields a total recession velocity $= 3099.2 \pm 150$\,\kms.
The final uncertainty in the total recession velocity is dominated by
systematic uncertainties in the 2M++ model. We also note that the 2M++
estimate is consistent, to within $\sim$5\%, with the Virgo and Great
Attractor infall models of \citet{Mould00}. Adopting $H_0 =
73$\,\kms\,Mpc$^{-1}$, we estimate the distance to NGC\,4441 to be $42.5 \pm
2.1$\,Mpc, corresponding to a distance modulus of $\mu = 33.14 \pm 0.11$\,mag,
where the uncertainty on $\mu$ is dominated by the uncertainty in the peculiar
velocity correction. We hereafter adopt 33.14\,mag as the distance modulus to
NGC\,4441.\footnote{\citet{Tully13} estimate a significantly smaller distance
to NGC\,4441 ($\mu = 31.43 \pm 0.14$\,mag; $D = 19.0$\,Mpc) based on surface
brightness fluctuation measurements from \citet{Tonry01}. If NGC\,4441 is at
this distance, then \sn\ peaks at $M_g \approx -16.8$\,mag, which is
significantly underluminous for a SN\,Ia. Given that \sn\ has a normal rise
time $t_\mathrm{rise} \approx 18$\,d (\S\ref{sec:phot}), relatively normal
spectra at peak (\S\ref{sec:spec}), and lacks the spectral signatures of
intrinsically faint SNe\,Ia (\S\ref{sec:spec_comp}), it is highly unlikely
that it is a factor of $\sim$10 less luminous in the \gztf\ filter than normal
SNe\,Ia. We therefore adopt the larger kinematic distance to NGC\,4441.}
% \todo{Lots of different catalogs provide
% metallicity for SDSS galaxies, should we report on these results at all?
% Port Z = 0.04 and Granada Z ~0.01, and do not agree, firefly ~ 0.6 0r 0.2
% (solar?) depending on weighting, probably not a great idea} \fromkate{that's
% a reasonable spread in values but likely different calibrations. I don't
% know which one is best or the references for comparing this value to. }

We estimate the total reddening towards \sn\ to be small. There is relatively
little line of sight extinction due to the Milky Way, $E(B-V) \approx
0.018$\,mag \citep{Schlafly11, Schlegel98}. Furthermore, we do not find
significant evidence for strong extinction in NGC\,4441. Figure~\ref{fig:NaD}
highlights the \ion{Na}{I} D absorption in the spectrum of \sn\ due to gas in
NGC\,4441 and the Milky Way from our highest-resolution spectrum, $R \approx
4000$, obtained with Binospec+MMT. The \ion{Na}{I} D absorption is weak, and
we estimate a total equivalent width (EW) $= 390$\,m\AA\ for NGC\,4441 and
$220$\,m\AA\ for the Milky Way. There is a systematic uncertainty of
$\sim$10\% on each of these estimates due to uncertainties in the
continuum-fitting procedure. Assuming similar properties for the dust in
NGC\,4441 and the Milky Way, we scale the color excess measurement for the
Milky Way by the ratio of \ion{Na}{I} D EWs to estimate $E(B-V) \approx
0.032$\,mag for \sn\ due to absorption in NGC\,4441. This yields a total color
excess towards \sn\ of $E(B-V) \approx 0.05$\,mag, which we adopt for the
subsequent analysis in this study. We note that this estimate is consistent,
to within the uncertainties, with the EW(\ion{Na}{I} D)--$E(B-V)$ relations
presented in \citet{Poznanski12}. Further supporting the claim of low
extinction is the lack of a detection of the \ion{K}{I} $\lambda\lambda$7665,
7699 interstellar lines or the diffuse interstellar band at 5780\,\AA, which
also serve as proxies for extinction (e.g., \citealt{Phillips13}).
% \frommb{Maybe worth noting how estimate below would change for peculiar dust
% as that seen in some SNe (e.g. SN 2014J, Rv=1.4)}

\begin{figure}
    \centering
    \includegraphics[width=3.35in]{./figures/NaD.pdf}
    %
    \caption{Zoom-in on our moderate resolution ($R \approx 4000$)
    MMT+Binospec spectrum of \sn\ highlighting absorption due to \ion{Na}{I}
    D in the host galaxy, NGC\,4441 (blue solid line), and the Milky Way
    (thin black line). The velocity scale is centered on the D$_1$ line in
    NGC\,4441, with the SDSS redshift shown via the vertical dashed line. No
    shift has been applied to the Milky Way lines. The \ion{Na}{I} D lines,
    which serve as a proxy for dust-obscuration along the line of sight
    (e.g., \citealt{Poznanski12,Phillips13}) are weak, indicating a
    relatively small amount of reddening.}
    %
    \label{fig:NaD}
\end{figure}



\section{Photometric Analysis}\label{sec:phot}

\subsection{The Time of First Light, \tfl}\label{sec:t_fl}

We estimate the time of first light, \tfl, for \sn\ following the procedure
described in \citet{Miller20}. Briefly, \citet{Miller20} model the early
emission from a SN\,Ia as a power-law in time, $f \propto (t -
t_\mathrm{fl})^\alpha$, where $f$ is the flux, $t$ is time, and $\alpha$ is
the power-law index. \tfl\ is assumed to be the same everywhere in the
optical, allowing us to simultaneously fit observations in each of the ZTF
filters.

An important caveat for \sn\ is that the observed early decline in the light
curve clearly does not follow the simple power-law model, and thus these
observations must be masked when performing the fit. We conservatively exclude
observations from the first two nights of ZTF detection from the fit (this
choice is conservative as it is unclear when the mechanism that powers the
initial bump in \sn\ no longer significantly contributes to the flux in the
\gztf\ and \rztf\ filters). From the fit we find \tfl$ = -17.5
\pm^{1.0}_{1.3}$\,d relative to \tbmax. We know that the time of explosion
must be $< -17.4$\,d based on the discovery detection in \citealt{Itagaki19},
and, by definition $t_\mathrm{fl} \ge t_\mathrm{exp}$, meaning a portion of
the posterior distribution for our model cannot be correct. We also find
$\alpha_g = 2.15 \pm^{0.49}_{0.36}$ and $\alpha_r = 1.91 \pm^{0.42}_{0.31}$.
These values are typical of the normal SNe Ia studied in \citet{Miller20}. If
we only exclude the first observation from the model fit we find significantly
different results with a rise time that increases by $\sim$5\,d and power-law
indices that increase by $\gtrsim 1$. We note that such a long rise is
unlikely, however, as our spectroscopic models (see \S\ref{sec:tardis})
estimate the time of explosion, $t_\mathrm{exp}$, to be $\sim$17.9\,d prior to
\tbmax, fully consistent with our estimate of \tfl.

% \fromkate{The rise time is long for its magnitude and stretch. From
% \citet{Gonzalez-Gaitan12}, the risetime would suggest a high stretch event,
% their fig. 12. Likely just the impact of the early component but worth
% stressing. }

\subsection{Luminosity of the Initial UV/optical Flash}\label{sec:luminosity}

To estimate the luminosity and temperature of the initial UV/optical flash from
\sn, we model the broadband spectral energy distribution (SED) as a blackbody.
The assumed distance and reddening towards \sn\ are taken from
\S\ref{sec:host}. The ZTF optical and \textit{Swift} UV observations were not
simultaneous, so we therefore interpolate the optical light curves to estimate
the flux during the same epochs as \textit{Swift} observations. While SNe\,Ia
do not emit as pure blackbodies, this assumption is reasonable for the early
flash from \sn, which is distinctly different from normal SNe. Furthermore,
our initial spectrum of \sn\ shows a blue and nearly featureless continuum
largely consistent with blackbody emission.

Following interpolation to an epoch 1.24~rest-frame~d after \tfl\
($\mathrm{JD} = $2,458,847.43), we estimate a blackbody luminosity $L = (1.7
\pm ^{0.2}_{0.1}) \times 10^{42}$\,erg\,s$^{-1}$ and temperature
$T_\mathrm{eff} = 14.8 \pm^{0.9}_{1.2}$\,kK. This estimate represents a lower
limit to the peak luminosity of the initial flash, as the UV flux was already
decreasing at this time (\textit{Swift} obtained two sets of UV observations
separated by $\sim$90\,min during the first epoch of observations, and the
$uvm2$ and $uvm1$ flux is clearly decreasing during this time; see
Figure~\ref{fig:p48}). iPTF\,14atg, the other SN\,Ia to exhibit an early UV
flash, has an estimated flash luminosity of $\sim$3$ \times
10^{41}$erg\,s$^{-1}$, a factor of $\sim$5 less than for \sn, though we
caution that a direct comparison of these estimates requires the
\textit{Swift} observations to have been acquired at precisely the same epoch,
which is likely not the case.

At an epoch 3.15~rest-frame~d after \tfl, we estimate the luminosity and
temperature to have fallen to $L = (7.0 \pm ^{0.9}_{0.6}) \times
10^{41}$\,erg\,s$^{-1}$ and $T_\mathrm{eff} = 8.7 \pm^{0.5}_{0.4}$\,kK,
respectively. For this epoch we have excluded the $uvw2$ flux from the
blackbody model due to the significantly lower temperature, and known red leak
for that filter (see \S\ref{sec:swift}). This measurement of $t_\mathrm{eff}$
is consistent with spectral modeling at a similar epoch (see
\S\ref{sec:tardis}). If we assume that the early flash peaked 1\,d after \tfl,
and effectively ended 3\,d after \tfl\ (note -- both of these assumptions are
highly uncertain), then the initial flash emitted a total integrated energy of
$\sim$4$\times 10^{42}$\,erg.
% peak ~ 1.82e42; total = 1.82e42/2 + 2*7e41 + 1.12e+42

\subsection{Maximum Light and Decline}\label{sec:max_decline}

While the rise time and power-law indices of \sn\ are similar to other normal
SNe Ia, the full photometric evolution does not resemble a normal SN\,Ia. The
photometric evolution of \sn\ is highlighted in Figure~\ref{fig:lc_comp},
where \sn\ is compared to 121 normal SNe\,Ia from \citet{Yao19}.\footnote{For
the purposes of this comparison we consider SN\,1991T-like, SN\,1999a-like,
and SN\,1986G-like events to all be normal SNe\,Ia.} \sn\ is somewhat
underluminous ($M_{g,\mathrm{max}} \approx -18.5$\,mag), declines rapidly
($\Delta m_{15}(g) = 1.30\pm^{0.01}_{0.02}$\,mag, uncertainties represent the
68\% credible region), and does not exhibit a ``shoulder'' in the \rztf\ or a
strong secondary maximum in the \iztf\ light curves post-maximum. The slightly
underluminous and moderately fast decline of \sn\ are very similar to the
SN\,1986G-like SNe\,Ia, which represent a transitional group between normal
SNe\,Ia and the underluminous SN\,1991bg-like class (e.g.,
\citealt{Taubenberger17} and references therein). While the photometric
evolution of \sn\ is similar to 86G-like SNe, we show that \sn\ is
spectroscopically distinct from these transitional SNe
(\S\ref{sec:spec_comp}).

\begin{figure}
    \centering
    \includegraphics[width=3.35in]{./figures/abs_mag_host_ebv_kcorr.pdf}
    %
    \caption{Photometric evolution of \sn\ compared to 121 normal SNe\,Ia
    observed by ZTF \citep{Yao19} in the \gztf\ (\textit{top}) and \rztf\
    (\textit{bottom}) filters. The normal SNe are shown as open grey circles,
    while the symbols for \sn\ are the same as Figure~\ref{fig:p48}. Relative
    to normal SNe\,Ia, \sn\ is fainter, declines faster in \gztf, and lacks
    the ``shoulder'' typically seen in the \rztf\ filter. Normal SNe light
    curves have been corrected for host-galaxy reddening and $K$-corrections
    have been applied, with both determined via \texttt{SNooPY} (see
    \citealt{Bulla20} for details of our implementation). $K$-corrections have
    not been applied to the light curve of \sn.}
    %
    \label{fig:lc_comp}
\end{figure}

% For context, of the 127 SNe\,Ia observed by ZTF and studied in \citet{Yao19}, only 1, ZTF18abclfee (SN\,2018crl), a SN\,2002cx-like event, had a faster decline than \sn. and, according to the new photometric classification scheme presented in \citet{Ashall20}, \sn\ is consistent with  91bg-like SNe.

We also find that standard SN\,Ia fitting techniques do not provide good
matches to the evolution of \sn. For example, a \texttt{SNooPY}
\citep{Burns11} fit to the optical light curve requires significant
host-galaxy extinction ($E(B-V)_\mathrm{host}\approx0.4$\,mag, cf.\
\S\ref{sec:host}) to match the observed red colors, while predicting a
secondary maximum in the \iztf-band and a fast evolution after peak that is
not seen in \sn. A \texttt{SALT2} \citep{Guy07} fit leads to similar
inconsistencies to those in \texttt{SNooPy}. These inconsistencies support our
conclusion above that the photometric evolution of \sn\ does not match normal
SNe\,Ia.

\subsection{Color Evolution}

\sn\ is further distinguished from normal SNe Ia by its unusual color
evolution (Figure~\ref{fig:colors}). The top panel of Figure~\ref{fig:colors}
shows the \gztf$ - $\rztf\ evolution of 35 normal SNe Ia with ZTF observations
within 3\,d of \tfl\ (see \citealt{Bulla20}), with the color evolution of \sn\
over-plotted. \sn\ exhibits a prominent blue to red to blue, or ``red bump,''
evolution in the $\sim$week after \tfl. Similar red bumps are only seen in
$\sim$10\% of the ZTF sample \citep{Bulla20}. Furthermore, while normal SNe Ia
exhibit a large scatter in \gztf$ - $\rztf\ shortly after \tfl\ they evolve to
form a tight locus around \tbmax. \sn\ is redder at peak than each of the
normal SNe Ia in the \citet{Bulla20} sample, and it exhibits a far more rapid
decline in \gztf$ - $\rztf. In this way, \sn\ is again intermediate between
normal SNe\,Ia and underluminous 91bg-like SNe. Figure~\ref{fig:colors} shows
that normal SNe\,Ia are reddest at $\sim$+30\,d, while 91bg-like SNe are
reddest between $\sim$+10--15\,d \citet{Burns14}. \sn\ reaches a
\gztf$-$\rztf\ maximum at an intermediate time of $\sim$+20\,d.

\begin{figure}
    \centering
    \includegraphics[width=3.35in]{./figures/P48_colors.pdf}
    %
    \caption{Photometric color evolution of \sn. \textit{Top}: \gztf$ -
    $\rztf\ evolution of \sn\ (solid green squares), corrected for the total
    line of sight extinction (see \S\ref{sec:host}), and compared with the
    evolution of 65 normal SNe Ia (open circles) observed within 3\,d of \tfl\
    by ZTF (from \citealt{Bulla20}). \sn\ is the reddest SN\,In the group, and
    it exhibits the fastest evolution to red colors post-\tbmax.
    \textit{Bottom}: the $uvm2 - uvw1$ (purple crosses), \gztf$ - $\rztf\
    (solid, green squares), and \rztf$ - $\iztf\ (open, red squares) color
    evolution of \sn. The ``red bump'' can clearly be seen in the optical.}
    %
    \label{fig:colors}
\end{figure}

The offset in the \gztf$ - $\rztf\ color evolution of \sn\ relative to normal
SNe\,Ia would be reduced if the reddening towards \sn\ has been significantly
underestimated. A color excess of $E(B-V) \approx 0.25$\,mag, rather than the
0.05\,mag adopted in \S\ref{sec:host}, would roughly align the pre-\tbmax\
\gztf$ - $\rztf\ color of \sn\ with the tight locus seen in
Figure~\ref{fig:colors}. Such a correction would also bring the peak optical
brightness of \sn\ in line with normal SNe\,Ia [for $E(B-V) \approx
0.25$\,mag, $M_g \approx -19.25$\,mag and $M_r \approx -19.1$\,mag for \sn].

While the spectral appearance of \sn\ is similar to some normal SNe\,Ia (see
\S\ref{sec:spec_comp}), the observed rapid decline in the \gztf\ filter
provides strong evidence that \sn\ is not a normal luminosity SN\,Ia.
\citet{Phillips93} showed that in the optical SNe\,Ia follow a
brightness--width relation, whereby brighter explosions decline less rapidly.
Thus, with a typical peak in the optical, as would be implied with $E(B-V)
\approx 0.25$\,mag, the fast decline in \sn\ [$\Delta m_{15}(g) = 1.3$] would
be largely unprecedented.\footnote{Only 2 normal SNe\,Ia in the \citet{Yao19}
sample decline faster than \sn\ as measured by $\Delta m_{15}(g)$. While the
lack of \textit{Swift} $b$-band templates prevents us from measuring $\Delta
m_{15}(B)$, the relationship between that and $\Delta m_{15}(g)$ for normal
ZTF SNe\,Ia suggests $\Delta m_{15}(B) \gtrsim 1.6$\,mag for \sn.} We therefore
conclude that the color excess towards \sn\ is not underestimated, and that
the SN\,Is instead intrinsically red in the optical.

% Such a large reddening would dramatically change the appearance of \sn\ in
% the UV. If we assume $E(B-V) \approx 0.25$\,mag, then the luminosity of the
% initial flash would be $\sim$8.7$\times 10^{42}$\,erg\,s$^{-1}$, more than a
% factor of 5 higher than what we estimated in \S\ref{sec:luminosity}.

% The bottom panel of Figure~\ref{fig:colors} shows the $uvm2 - uvw1$
% color evolution of \sn. At \tbmax, $uvm2 - uvw1 \approx 1.2$\,mag, making
% \sn\ one of the most UV blue SNe\,Ia at maximum light (e.g.,
% \citealt{Milne10,Brown17}). A color excess of $E(B-V) \approx 0.25$\,mag is
% equivalent to $E(uvm2 - uvw1) \approx 0.5$\,mag using the Fitzpatrick+
% (1999) extinction law, $R_V = 3.1$, and the effective wavelengths of the
% \textit{Swift} filters from \citet{Brown17} \frommb{I get 0.65 mag, maybe
% double-check?}. \fromkate{I did this calc, I've checked and using method and
% values listed, I get 0.5 mag for the uvm2-uvw1 colour} This results in a
% $uvm2 - uvw1 \approx 0.XXX$\,mag at \tbmax, making \sn\ the most UV blue
% SN\,Ia observed by \textit{Swift} \fromkate{this needs to be updated because
% none of the objects in Brown sample are corrected for extinction}.

Even if one ignores the striking initial bump in the light curve of \sn, we
can still conclude that \sn\ is not a normal SN\,Ia based on its other
photometric properties (e.g., relatively faint peak optical brightness,
moderately fast decline, lack of a near-infrared secondary maximum, and red
appearance at peak).

\section{Spectral Evolution of \sn}\label{sec:spec}

Optical spectra of \sn\ were obtained at phases from $-$14.9\,d (2.6\,d after
\tfl) to 66.5\,d after \tbmax. Details of the spectra are presented in
Table~\ref{tab:spectra} and the spectral evolution is shown in
Figure~\ref{fig:spec_evo}. The absorption features in \sn\ are typical of
SNe\,Ia, including intermediate mass elements (IMEs), primarily Si, Ca, and O,
as well as iron-group elements (IGEs).

\subsection{TARDIS Models}\label{sec:tardis}

To determine the structure of the ejecta and relative contributions of
different ions at early and maximum light phases, we have modeled the spectra
at $-$14.9\,d, $-12.0$\,d, and $+$0.0\,d using the 1D Monte Carlo radiative
transfer code \texttt{TARDIS} \citep{Kerzendorf14}. We note that
\texttt{TARDIS} assumes a single, sharp photosphere between the optically
thick and thin regions. Therefore, if there is a contribution to the spectrum
from an underlying quasi-blackbody (at early times this could be due to
interaction, for example; see Sect. XX), this will impact the ability of
\texttt{TARDIS} to fully reproduce the observations. Nevertheless, our models
should provide a reasonable approximation of the plasma state within the
ejecta. Parameters of our \texttt{TARDIS} models are given in
Table~\ref{tab:tardis}.

\begin{deluxetable*}{lcrrccc}
\tabletypesize{\scriptsize}
\tablewidth{0pt}
\tablecaption{\texttt{TARDIS} input parameters\label{tab:tardis}}
\tablehead{
\colhead{Date} &
\colhead{MJD} &
\colhead{Phase\tablenotemark{a}} &
\colhead{$t - t_\mathrm{exp}$\tablenotemark{b}} &
\colhead{$L$\tablenotemark{c}} &
\colhead{$v_\mathrm{boundary}$\tablenotemark{d} }&
\colhead{$T_\mathrm{boundary}$\tablenotemark{e} } \\
\colhead{(UT) }&
\colhead{} &
\colhead{(d)} &
\colhead{(d)} &
\colhead{($\log L_{\odot}$)} &
\colhead{(\kms) } &
\colhead{(K)}
}
\startdata
2019 Dec 31.277 & 58848.277 & $-14.9$ & 3.0 & 8.55 & 25,000 & 8173 \\
2020 Jan 03.217 & 58851.217 & $-12.0$ & 6.0 & 8.60 & 16,500 & 7925 \\
2020 Jan 15.392 & 58863.392 & $+0.0$ & 18.0 & 9.29 & 10,500 & 9696 \\
\enddata
\tablenotetext{a}{Rest-frame time relative to the time of $B$-band maximum,
\tbmax.}
\tablenotetext{b}{Rest-frame time relative to the \texttt{TARDIS} time of
explosion, $t_\mathrm{exp}$.}
\tablenotetext{c}{Emergent Luminosity.}
\tablenotetext{d}{Ejecta velocity at the inner boundary of the photosphere.}
\tablenotetext{e}{Temperature at the inner boundary of the photosphere. The
inner boundary temperature is not explicitly an input parameter for
\texttt{TARDIS}, it is derived from the luminosity, time since explosion, and
inner boundary velocity.}

\end{deluxetable*}

%%%Description of overall modelling results, velocities, features present, temperature. Then detailed discussion of Mg II vs Si III can be removed.
The first spectrum of \sn\ at $-$14.9\,d (2.6\,d after \tfl, 3.0\,d after
explosion) shows shallow features consistent with IMEs moving at extremely
high velocities ($>$\,20,000\,\kms, Figure~\ref{fig:spec_evo}). The
best-fitting \texttt{TARDIS} model is shown in Figure~\ref{fig:tardis}, along
with the contribution of individual elements to the spectral features. For
this model, we have assumed a uniform composition of O, Mg, Si, and S. Our
model demonstrates that the shallow absorption features observed at this phase
can be reproduced solely by IMEs (predominantly \ion{Si}{II}), and that the
presence of IGE is not required to match the data. Our model also confirms the
high velocities of the ejecta -- we find the spectral features and temperature
are best reproduced with a photospheric velocity of
$\sim$25,000\,\kms.

\begin{figure*}
    \centering
    \includegraphics[width=\textwidth]{./figures/tardis.pdf}
    %
    \caption{Comparison of \texttt{TARDIS} models to \sn\ at $-$14.9\,d
    (left), $-12.0$\,d (middle), and $+$0.0\,d (right), relative to \tbmax.
    For each model, we color code a histogram showing the contribution of
    each element to the spectroscopic features, based on the last element with
    which a Monte Carlo packet experienced an interaction. Packets may be
    absorbed and re-emitted at different wavelengths, with the exception of
    those packets that only experience electron scattering. During electron
    scattering, only the direction of propagation is changed. These packets
    are shown in grey. Packets that did not interact during the simulation are
    shown in black. Contributions above/below zero demonstrate the sum of
    packet luminosities after/before their last interaction.}
    %
    \label{fig:tardis}
\end{figure*}

Similarly, for the $-12.0$\,d spectrum we find that a model that does not
contain IGE above $\sim$16,500\,\kms\ reproduces the majority of the
spectroscopic features. Again, our model contains a uniform composition of O,
Mg, Si, and S, and is shown in Fig.~\ref{fig:tardis}. At this phase the model
suggests the photospheric temperature has not significantly changed, however
the features have become much more prominent. Compared to modeling of the
spectroscopically similar SN\,2002bo \citep{Stehle05} at the same epoch, see
below, we find \sn\ has a lower photospheric temperature ($\sim$8,000\,K,
compared to $\sim$9,500\,K for SN\,2002bo).

While the early spectra of \sn\ are dominated by IMEs, we do not find evidence
for \ion{C}{II} absorption. Even increasing the carbon abundance to relatively
large amounts (50\%), we do not find strong \ion{C}{II} features produced in
our earliest spectra. The presence of notable \ion{C}{II} features is likely
affected by the extremely high velocities of \sn, causing blends with
\ion{Si}{ii}. Therefore we are unable to place meaningful constraints on the
carbon abundance in the very outermost ejecta.

Given that our maximum light spectrum occurs 12 days after our previous model
spectrum, we assume a composition for the inner ejecta
($\textless$16,500\,\kms\ ) similar to that found for SN\,2002bo
\citep{Stehle05}. A more detailed ejecta structure could be achieved through
modeling additional pre-maximum spectra, but is beyond the scope of the work
presented here. As shown in Fig.~\ref{fig:tardis}, our model is able to
broadly reproduce many of the features observed. Notable exceptions include
the features around $\sim$4200 and 4900\,\AA, which we attribute to Fe.
Better spectroscopic agreement could potentially be achieved if \sn\ had
a lower abundance of IGE within the inner ejecta relative to SN\,2002bo.

Overall, our \texttt{TARDIS} modeling demonstrates that \sn\ is consistent
with a low (or zero) fraction of IGE in the outer ejecta. Additionally, the
earliest phases show little change in temperature, as expected from the
color evolution.

\subsection{\ion{Si}{II} Evolution}\label{sec:SiII}

We have measured the velocity of the \ion{Si}{II} $\lambda$6355 absorption
feature following the procedure in \citet[][see their \S2.5]{Maguire14}. We
have also estimated the pseudo-equivalent widths (pEWs) of the \ion{Si}{II}
$\lambda\lambda$5972, 6355 features, allowing us to measure their ratio, also
known as $\mathcal{R}($\ion{Si}{II}$)$; see \citet{Hachinger08} for the
updated definition relative to \citet{Nugent95}.

\begin{figure}
    \centering
    \includegraphics[width=3.35in]{./figures/vel_evolution.pdf}
    %
    \caption{Velocity evolution of \ion{Si}{II} $\lambda$6355 absorption in
    \sn\ (large, filled circles). For comparison we also show the measurements
    for 264 SNe Ia observed by the Palomar Transient Factory \citep[PTF; data
    from][]{Maguire14} as open circles, with SN\,2010jn (PTF\,10ygu), the SN
    with the fastest moving ejecta in the PTF sample, highlighted via orange
    crosses. We additionally show the velocity evolution of SN\,2002bo
    \citep[data from][]{Benetti04}, a SN that is very similar to \sn, as open
    diamonds. The median velocity evolution of each of the spectroscopic
    classes defined by \citet[][Shallow Silicon, Core Normal, Broad Line, and
    Cool]{Branch06} are shown via solid lines. It is clear that \sn\ has
    exceptionally high-velocity ejecta relative to typical SNe Ia.}
    %
    \label{fig:vel_evo}
\end{figure}

%%%Velocity and equivalent width measurements?
The velocity evolution of \ion{Si}{II} $\lambda$6355 is shown in
Figure~\ref{fig:vel_evo}, compared to measurements for the Palomar Transient
Factory (PTF) SN\,Ia sample from \citet{Maguire14} and the median velocity
evolution of SNe\,Ia belonging to the four different classes (Shallow
Silicon, Core Normal, Broad Line, and Cool) identified in
\citet{Branch06};\footnote{The velocity measurements are from
\citet{Blondin12}, while the method to determine the median velocity is
described in \citet{Miller18}.} hereafter, the \citeauthor{Branch06}~class.
The \ion{Si}{II} $\lambda$6355 velocity is $\sim$15000 \kms.

At \tbmax, the pEW measurements for the \ion{Si}{II} $\lambda$6355 and
$\lambda$5972 features are $183\pm1$\,\AA, and $13\pm2$\,\AA, respectively,
unambiguously classifying \sn\ as a \citeauthor{Branch06}~Broad Line SN\,Ia.
\sn\ stands out in Figure~\ref{fig:vel_evo} with some of the highest
\ion{Si}{II} velocities that have ever been observed. Within the PTF sample,
only SN\,2010jn (PTF\,10ygu) exhibits a \ion{Si}{II} absorption velocity as
high as \sn\ at every phase in its evolution.

As first noted by \citet{Nugent95}, and later confirmed by
\citet{Hachinger08}, $\mathcal{R}($\ion{Si}{II}$)$ is a luminosity indicator,
with larger values of $\mathcal{R}($\ion{Si}{II}$)$ corresponding to lower
luminosities. This correlation is driven by the ionization balance of
\ion{Si}{II}/\ion{Si}{III}, with cooler objects having stronger \ion{Si}{II}
$\lambda$5972 features. In Figure~\ref{fig:r_evo}, we show the
$\mathcal{R}($\ion{Si}{II}$)$ of \sn\ as a function of time, compared to
SN\,2011fe, SN\,2002bo and 5 SNe, that have multiple measurements over a long
baseline, from the PTF SN\,Ia spectral sample. Figure~\ref{fig:r_evo} shows
that most SNe\,Ia have a relatively flat evolution in
$\mathcal{R}($\ion{Si}{II}$)$ in the time leading up to \tbmax\ \citep[see
also][]{Riess98a}. \sn\ and SN\,2002bo, however, feature a very different
evolution with initially large values of $\mathcal{R}($\ion{Si}{II}$)$ that
rapidly decrease to very low values between $\sim$10 and $\sim$5\,d before
\tbmax.

\begin{figure}
    \centering
    \includegraphics[width=3.35in]{./figures/R_evolution.pdf}
    %
    \caption{Evolution of the ratio of the pEW of \ion{Si}{II} $\lambda$5972
    to \ion{Si}{II} $\lambda$6355, $\mathcal{R}($\ion{Si}{II}$)$, in \sn\
    (large, filled circles). SN\,2002bo \citep[data from][]{Benetti04} and
    SN\,2011fe \citep[data from][]{Pereira13} are also highlighted as open
    diamonds and open squares, respectively. For comparison we also show the
    $\mathcal{R}($\ion{Si}{II}$)$ evolution for 5 PTF SNe\,Ia (10mwb, 10qjq,
    10tce, 10wof, 11hub) with $> 3$ measurements over a duration $> 8$\,d
    (data from \citealt{Maguire14}) and SN\,2017erp \citep[data
    from][]{Brown19} as connected, open circles. \sn\ and SN\,2002bo exhibit
    an unusual inversion in $\mathcal{R}($\ion{Si}{II}$)$ as they evolve
    toward maximum light.}
    %
    \label{fig:r_evo}
\end{figure}

At face value, the $\mathcal{R}($\ion{Si}{II}$)$ evolution in
Figure~\ref{fig:r_evo} suggests that the effective temperature of \sn\
increases significantly as it rises to maximum light. The optical colors (see
Figure~\ref{fig:colors}), however, are nearly constant throughout this phase
while the UV$ - $optical colors clearly decline during the same period,
suggesting a decline in the effective temperature. The \texttt{TARDIS}
modeling also does not require a significant increase in temperature towards
maximum light from the temperature of $\sim$8,000\,K required at early times.
This stands in contrast to SN\,2002bo, which increases in temperature from
$\sim$9,500\,K at $-12.9$\,d to $\sim$14,000\,K at maximum light
\citep{Stehle05}. This temperature is similar to \citeauthor{Branch06}~Core
Normal SNe, such as SN\,2011fe, which typically have temperatures of
$\sim$14,500--15,000\,K at maximum light \citep{Mazzali14}.

\citet{Benetti04} interpreted these competing effects as the result of
significant \ion{Si}{II} mixing in the ejecta of SN\,2002bo. Mixing or Si
production in the outermost layers of the ejecta would (i) lead to larger
\ion{Si}{II} velocities, (ii) produce \ion{Si}{II} line ratios that indicate
cool temperatures (because there is less radioactive material to heat the
ejecta), before eventually (iii) producing low values of
$\mathcal{R}($\ion{Si}{II}$)$ as the photosphere recedes through the ejecta
to higher temperature regions. This picture is consistent with the
\citet{Stehle05} models of SN\,2002bo. In those models, Si completely
dominates the species at velocities above $\sim$23,000\,\kms, while there is
very little ($\sim$1\%) IGE above a 1.35\,$M_\odot$ in radial mass
coordinates. The same explanation does not clearly apply to \sn, however, as
our \texttt{TARDIS} models do not show evidence for a significant increase
in the photospheric temperature at maximum light. Our \texttt{TARDIS} models
are, unlike SN\,2002bo, consistent with no IGEs in the outer layers of the
\sn\ ejecta. A possible explanation for the evolution of
$\mathcal{R}($\ion{Si}{II}$)$ in \sn\ is that as the photosphere recedes to
regions with more \radni, increased Si ionization occurs leading to a
relative decrease in \ion{Si}{II} $\lambda$5972 absorption despite
relatively small changes in the photospheric temperature.

\kate{and} \magee{This still seems like an unexplained puzzle to me...}

\subsection{Spectral Comparison}\label{sec:spec_comp}

%%%Comparison to other objects & Si II ratio
In Figure~\ref{fig:spec_comp}, we compare the spectral evolution of \sn\ to
two Broad-Line SNe, SN\,2002bo and SNe\,2010jn, and two Cool SNe, SN\,1986G
and SN\,2004eo \citep{Cristiani92,
Benetti04,Pastorello07,Silverman11,Hachinger13,Maguire14} at four phases,
pre-maximum, maximum, $\sim$1 week post maximum, and $\sim$6 weeks post
maximum. The evolution of \sn\ and SN\,2002bo is remarkably similar at all
phases. The only significant difference between the two is the absorption
trough at $\sim$4800\,\AA\ in the pre- and maximum-light spectra. This
feature, which is typically attributed to a combination of \ion{Fe}{II},
\ion{Fe}{III}, and \ion{Si}{II}, is extremely narrow in \sn. This is in
agreement with the \texttt{TARDIS} modeling results where no Fe is required
in the outer ejecta of \sn\ to match the observed spectra at early times.
SN\,2010jn, which exhibits large \ion{Si}{II} velocities like \sn, shows
weaker IME absorption and stronger IGE absorption than \sn. While the
\citeauthor{Branch06}~Cool SNe\,1986G and 2004eo feature lower velocities than
\sn, there is strong agreement in the relative \ion{Si}{II} line strengths of
SN\,1986G and the earliest spectra of \sn.

% The left panel of Figure~\ref{fig:spec_comp} also shows the spectrum of
% SN\,1986G, a \citet{Branch06} Cool SN that is sometimes referred to as
% ``transitional'' given its intermediate properties between normal SNe\,Ia and
% the sub-luminous SN\,1991bg-like population (e.g., \citealt{Pastorello07}).
% The \ion{Si}{II} line ratios, and narrow $\sim$4800\,\AA\ feature in SN\,1986G
% are similar to \sn, confirming the cool nature of the photosphere at this
% early epoch. The $-12.0$\,d spectrum of \sn\ additionally shows a weaker blue
% line in the \ion{S}{II} ``W'' absorption feature at $\sim$5400\,\AA, which is
% also consistent with cool temperatures \citep{Nugent95}.

\begin{figure*}
    \centering
    \includegraphics[width=7.25in]{./figures/spec_comp_extinction.pdf}
    %
    \caption{Spectral comparison of \sn\ to \citeauthor{Branch06}~Broad Line
    and Cool SNe\,Ia. All spectra have been corrected for the total
    line-of-sight extinction with adopted $E(B-V)$ values of 0.9\,mag,
    0.38\,mag, 0.39\,mag, 0.109\,mag, and 0.05\,mag for SNe\,1986G
    \citep{Phillips87}, 2002bo \citep{Stehle05}, 2010jn \citep{Hachinger13},
    2004eo \citep{Pastorello07}, and \sn\ (this work), respectively.
    \textit{Left panel}: pre-maximum spectra showing the similarity of \sn\
    and SN\,2002bo. While the expansion velocities in the Cool SN\,1986G
    spectrum are considerably lower than those in the Broad Line SNe, the
    relative ratios of the \ion{Si}{II} features are similar to \sn.
    \textit{Second panel}: Comparison of \sn\ to the Broad Line SNe\,2002bo
    and SN\,2010jn. These SNe all feature nearly identical maximum-light
    spectra. By this phase, the relative strength of the \ion{Si}{II}
    absorption features is no longer similar to \citet{Branch06} Cool SNe, as
    illustrated by SN\,2004eo. \textit{Third panel}: $\sim$1 week post-maximum
    spectra. \textit{Fourth panel}: Transitional phase spectra. Comparison
    spectra have been downloaded from WISeREP \citep{Yaron12}, with spectra
    for individual SNe from the following sources: SN\,1986G --
    \citet{Cristiani92}, SN\,2002bo -- \citet{Benetti04,Silverman11},
    SN\,2010jn (PTF\,10ygu) -- \citet{Hachinger13,Maguire14}, SN\,2004eo --
    \citet{Pastorello07}.}
    %
    \label{fig:spec_comp}
\end{figure*}

The maximum-light spectra shown in the second panel of
Figure~\ref{fig:spec_comp} reveal a much higher luminosity/temperature for
\sn, as the \ion{Si}{II} $\lambda$5972\,\AA\ absorption has nearly disappeared
around \tbmax\ (see discussion of $\mathcal{R}($\ion{Si}{II}$)$ in
\S\ref{sec:SiII}). The appearance of \sn, SN\,2002bo, and SN\,2010jn are all
similar at this epoch, with the exception of the 4800\,\AA\ feature mentioned
above. SN\,2004eo has a similar appearance to \sn, though it has lower
velocities and cooler temperatures (as traced by \ion{Si}{II} $\lambda$5972).

The $+9.2$\,d spectrum of \sn, shown in the third panel of
Figure~\ref{fig:spec_comp}, shows absorption due to IGE. Additional
differences between \sn\ and SN\,2002bo can be seen at this phase. There is
stronger absorption in \sn\ blueward of \ion{Ca}{II} H\&K, and the \ion{S}{II}
``W'' absorption feature is still present in \sn\ and it cannot be identified
in SN\,2002bo or SN\,2010jn. SN\,2004eo maintains an appearance that is
somewhat similar to \sn, though as before, the temperatures are cooler and the
velocities lower.

Spectra obtained $\sim$6 weeks after maximum light are shown in the fourth
panel of Figure~\ref{fig:spec_comp}. By this time, as the SNe are
transitioning into a nebular phase, the appearance of each spectrum is similar
modulo some minor differences in the relative line strengths of different
features.

\kate{do you want to say anything about \ion{Ca}{II}?}

\section{A Physical Explanation for \sn}\label{sec:models}

The most striking feature of \sn\ is the observed UV/optical peak that occurs
shortly after discovery (Figure~\ref{fig:p48}). Any model to explain \sn\ must
account for this highly unusual feature. A UV decline in the early phase of a
SN\,Ia has previously only been observed in a single event, iPTF\,14atg
\citep{Cao15}. Clearly resolved ``bumps'' in the early optical emission of
SNe\,Ia are also rare, having only been seen in a few events: SN\,2017cbv
\citep{Hosseinzadeh17} and SN\,2018oh \citep{Shappee19,Dimitriadis19}.

\sn\ features other properties, in addition to an initial peak $\sim$17\,d
prior to \tbmax, that separate it from normal SNe\,Ia. A good model should be
able to explain the properties:
%
\begin{enumerate}
    \item The early UV/optical ``flash'' (Figure~\ref{fig:p48}).
    \item The moderately faint peak in the optical (\S\ref{sec:max_decline}). 
    \item The relatively fast optical decline (\S\ref{sec:max_decline}). 
    \item The red optical colors at all epochs (Figure~\ref{fig:colors}). 
    \item The lack of IGE in the early spectra (\S\ref{sec:tardis}).
    \item The evolution in \RSiII\ (\S\ref{sec:SiII} and Figure~\ref{fig:r_evo}).
    \item The large photospheric velocities (Figure~\ref{fig:vel_evo}).
    % \item The lack of C in the early spectra (\S\ref{sec:tardis}) \todo{check all discussions of Carbon}
\end{enumerate}
%
The moderately faint peak combined with the large \ion{Si}{II} velocity is
particularly rare (see e.g., Figure~11 in \citealt{Polin19}, where \sn\ would
be well isolated from all the other SNe\,Ia). This suggests that \sn\ has a
relatively high kinetic energy despite a relatively low \radni\ yield.

As noted in \S\ref{sec:max_decline}, the photometric evolution of \sn\ is
similar to intermediate 86G-like SNe, however, the spectra feature much weaker
\ion{Si}{II} $\lambda$5972 absorption and larger expansion velocities than
what is seen in 86G-like SNe (see Figure~\ref{fig:spec_comp}). Similarly,
while the spectral appearance and evolution of \sn\ is similar to SN\,2002bo,
and other \citeauthor{Branch06}~Broad Line SNe, the photometric properties are
entirely different. SN\,2002bo features a relatively slow decline
[$\Delta{m}_{15}(B) = 1.13$\,mag] with a clear secondary maximum in the $I$
band \citep{Benetti04}, which stands in contrast to what is observed in \sn.

If we otherwise ignore the early flash, several of the remaining features
(2--6) in the list above can be understood if the explosion that gave rise to
\sn\ produced a relatively small amount of \radni\ that is strongly confined
to the inner regions of the SN ejecta. A low \radni\ yield could explain the
underluminous light curve and red colors, while a highly stratified ejecta
structure could explain the lack of IGE in the early spectra, as the IGE would
not have been mixed to these outer layers. Furthermore, with a highly
stratified ejecta composition, the photosphere would transition somewhat
rapidly from \radni-poor to \radni-rich, resulting in a significant change in
the luminosity/temperature of the ejecta along the lines of what we see in the
evolution of \RSiII.

\citet{Magee20} developed a suite of models featuring different \radni\
structures within the SN ejecta. These models were compared to early
observations of SNe\,Ia to see which ones replicate what is observed in
nature. Generally, it is found that highly stratified models do not match the
early evolution of normal SNe\,Ia \citep{Magee20}. However, when we model the
early evolution of \sn\ using the models of \citet{Magee20}, we find that the
observations are best matched by highly stratified models. For this modelling
we have excluded the first two epochs of ZTF observations, as we consider the
mechanism that produces the early UV flash to be different from the standard
\radni\ decay that powers most SNe\,Ia. That the (normal) rising portion of
the \sn\ light curve is best matched by stratified models strengthens the
support for this interpretation. We note, however, that \citet{Magee20}
demonstrate that the time of first detection can dramatically alter the
inferred model properties and it is unclear which epochs (if any) should be
excluded. Nevertheless, a stratified ejecta is also consistent with our
spectroscopic analysis (see \S\ref{sec:tardis}).

On their own, a low-\radni\ yield and highly stratified ejecta fail to explain
the blue UV/optical flash seen in \sn. A large number of scenarios have been
proposed to explain early ``bumps'' or ``flashes'' in SNe\,Ia light curves,
including: interaction between the SN ejecta and the WD binary companion
\citep{Kasen10a}, interaction between the SN ejecta and circumstellar material
(e.g., \citealt{Dessart14,Piro16,Levanon17}), shock cooling following the
shock breakout from the surface of the WD (e.g., \citealt{Piro10,Rabinak11}),
double detonation explosions (e.g., \citealt{Noebauer17,Polin19}), and
extended clumps of \radni\ in the SN ejecta (e.g.,
\citealt{Shappee19,Dimitriadis19}). We discuss these models and their ability
to replicate observations of \sn\ below.\footnote{We do not discuss shock
breakout models as our initial detection of \sn\ occurred at $M_g \approx
-16.3$\,mag. A progenitor radius of $\sim$10$\,R_\odot$ is needed to explain
such a high luminosity \citep{Piro10,Rabinak11}, which we consider implausible
for a WD.}

\subsection{Companion Interaction}

For SD progenitors of SNe\,Ia, the SN ejecta will shock on the surface of the
non-degenerate companion giving rise to a short-lived transient in the days
after explosion. \citet{Kasen10a} provided models for the appearance of this
interaction, which is primarily dependent upon the binary separation of the
system (assuming Roche lobe overflow for the non-degenerate companion). The
observed emission following the ejecta-companion collision is dependent upon
the orientation of the system at the time of explosion relative to the line of
sight \citep{Kasen10a}.

An analytic formulation for the luminosity and effective temperature of the
emission from the companion shock is given in Equations~22 and 25 in
\citet{Kasen10a}. \citet{Brown12} provide an analytic function to approximate
the fractional decrease in the observed flux as a function of the orientation
of the system. We combine equations from \citet{Kasen10a} and \citet{Brown12}
to model the early emission from \sn\ as ejecta-companion collision. We assume
the interaction emits as a blackbody, and that the electon scattering opacity
$\kappa_e = 0.2$\,cm$^{2}$\,g$^{-1}$ (as in \citealt{Kasen10a}). Assuming
$z_\mathrm{SN} = 0.0094$, $E(B-V)_\mathrm{MW} = 0.018$\,mag, and
$E(B-V)_\mathrm{host} = 0.032$\,mag, we compare observed flux measurements
with those predicted by the \citet{Kasen10a} model in epochs with MJD$\,<
58849.2$ (i.e., the first $\sim$2.5\,d after discovery when emission from the
companion interaction is significantly brighter than the luminosity due to
radioactive decay).\footnote{Given that \sn\ is an unusual SN, we make no
assumptions about the ``normal'' SN emission due to radioactive decay of
\radni. The companion-interaction model should therefore
\textit{underestimate} the observed flux as there will be a growing
contribution due to radioactive decay with time.} The model parameters,
including: the companion separation, $a$, the mass of the ejecta,
$M_\mathrm{ej}$, the velocity of the ejecta, $v_\mathrm{ej}$, the angle
between the observer, the SN, and the companion, $\theta$, and the time of
explosion, $t_\mathrm{exp}$ are constrained via a Gaussian likelihood and flat
priors (see Table~\ref{tab:priors}) using an affine-invariant
\citep{Goodman10} Markov Chain Monte Carlo (MCMC) ensemble sampler
\citep{Foreman-Mackey13}.

\begin{figure}
    \centering
    \includegraphics[width=3.35in]{./figures/sn_companion_models.pdf}
    %
    \caption{SN ejecta-companion interaction models compared with the
    UV/optical observations of \sn. Observation symbols are the same as
    Figure~\ref{fig:p48} (solid magenta squares show \textit{Swift} $uvw2$
    observations that are not shown in Figure~\ref{fig:p48}). Solid lines show
    companion interaction model predictions in each filter (the lines have the
    same colors as the corresponding symbols for each passband). The maximum a
    posteriori model is shown via the single bold lines, while other random
    draws from the posterior are shown as thin transparent lines. The shaded
    area shows observations that are excluded from the model fit. The
    overprediction of the optical flux $\sim$13.7\,d prior to \tbmax\ suggests
    that companion interaction does not explain the early flash in \sn\ (see
    text).}
    %
    \label{fig:companion}
\end{figure}

The results of this procedure are shown in Figure~\ref{fig:companion}, where
it is clear that the model presented in \citet{Kasen10a} does an adequate job
of explaining the early UV/optical emission from \sn. We find marginalized
posterior values of $a = 9.1 \pm 0.7 \times 10^{11}$\,cm, $M_\mathrm{ej} = 1.0
\pm 0.3\,M_\odot$, $v_\mathrm{ej} = 2.2 \pm^{0.5}_{0.3} \times 10^{4}$\,\kms,
$\theta = 34 \pm^{28}_{24} \deg$, and $t_\mathrm{exp}(\mathrm{MJD}) = 58845.83
\pm 0.05$ (all uncertainties are 68\% credible regions). Examination of a
corner plot of the posterior samples shows that $M_\mathrm{ej}$ is largely
unconstrained, while $v_\mathrm{ej}$ is degenerate with $\theta$ and $a$ is
degenerate with $t_\mathrm{exp}$. 

While the interaction models roughly approximate the SN emission in the
$\sim$3\,d after explosion, they significantly \textit{overestimate} the flux
immediately after the fitting window as shown in Figure~\ref{fig:companion}.
This problem is exacerbated by the fact that the models do not include
emission associated with radioactive decay, meaning the true discrepancy
between what is predicted and what is observed is even larger than what is
shown in Figure~\ref{fig:companion}. If we extend the fitting window to
include the optical observations obtained $\sim$13.75\,d before \tbmax, the
interaction models still overpredict the optical flux at this epoch. This
overprediction of the optical flux poses a challenge for the companion
interaction scenario; an inability to simultaneously match both UV and optical
observations has been noted for other SNe\,Ia with early bumps or linear rises
\citep{Hosseinzadeh17,Miller18}.

For this reason, we do not favor the ejecta-companion interaction
interpretation for \sn. \citet{Kasen10a} notes several assumptions and
approximations in the derivation of the equations used to estimate the
emission from the companion shock. It is possible that the inclusion of more
detailed physics, or additional complexity in the analytic formulation of the
models,\footnote{For example, \citet{Kasen10a} points out that the derived
equation for the luminosity of the shock interaction does not account for the
advected luminosity that would be seen in the observer frame.} could better
reconcile companion interaction models with \sn. Such improvements are beyond
the scope of this paper, leading us to explore other explanations for the
early flash.

\subsection{Ni Clumps in the SN Ejecta}

SN\,2018oh was observed with an exquisite 30\,min cadence by the
\textit{Kepler} spacecraft and showed a clearly delineated linear rise in
flux followed by a ``standard'' $t^2$ power-law $\sim$4\,d after \tfl. Models
with extended clumps of \radni\ just below the WD surface have been proposed
as a possible explanation for the initial linear rise in SN\,2018oh
\citep{Shappee19,Dimitriadis19}. The models considered in \citet{Shappee19}
and \citet{Dimitriadis19}, which build on the work of \citet{Piro16}, only
cover the first $\sim$10\,d after explosion and assume relatively simple grey
opacities. To further investigate this possibility, Magee \& Maguire (2020)
recently performed more detailed radiative transfer calculations for SNe\,Ia
with extended clumps of \radni. They then compared these models to SN\,2018oh
and SN\,2017cbv, another event with a clearly resolved bump in the early
light curve \citep{Hosseinzadeh17}.

\begin{figure}
    \centering
    \includegraphics[width=3.35in]{./figures/clump_spec.pdf}
    %
    \caption{Spectroscopic comparison between \sn\ and our model with a clump
    of \radni\ in the outer ejecta. Observed spectra of \sn\ are shown in
    blue, with phases marked relative to \tbmax, whereas the model spectra are
    shown in dark grey, with phases marked relative to the modelled time of
    explosion. For the comparison we have adopted a model explosion time
    $t_\mathrm{exp} = t_\mathrm{fl} - 1.67$\,d. The modelled spectra have been
    smoothed with a Savitzky-Golay filter \citep{Savitzky64}. While an
    extended clump of \radni\ in the SN ejecta can recreate the early optical
    flash, it leads to strong blanketing in the blue portion of the optical
    that is not observed around maximum light in \sn. }
    %
    \label{fig:Ni_bullet}
\end{figure}

For \sn\ we follow the procedure in Magee \& Maguire (2020) to model the early
flash and rise of the SN. Briefly, we exclude the first two epochs of optical
detections in ZTF, and identify the best-fit model to the later evolution of
the SN from the grid of models created in \citet{Magee20}. Following the
generation of this ``baseline'' model, we add clumps of \radni\ to the outer
layers of the SN ejecta, and perform full radiative transfer calculations
using \texttt{TURTLS} \citep{Magee18}. We find that a model with a
0.02~$M_{\rm{\odot}}$ clump of \radni\ adequately matches the early optical
evolution of \sn. As found in Magee \& Maguire (2020), however, such models
face a significant challenge in that the extended clump of \radni\
dramatically alters the appearance of the SN at maximum light.
Figure~\ref{fig:Ni_bullet} shows a comparison of the observed spectra with our
calculated models. The Ni clump models feature strong blanketing in the
blue-optical, which is simply not present in the observed spectra of \sn. We
therefore conclude that Ni clumps cannot explain the early flash seen in \sn.

\subsection{Double Detonation Models}

WDs that accrete a thin shell of He can explode via a ``double detonation''
whereby explosive burning in the He shell drives a shock into the C/O core of
the WD. This shock can ignite explosive C burning and a detonation that
disrupts the entire star (e.g., \citealt{Nomoto82,Nomoto82a,Woosley94}). Such
explosions are even possible in C/O WDs that are well below the Chandrasekhar
mass (see \citealt{Fink07, Fink10} and references therein).

Recent models of double detonation explosions presented in \citet{Polin19}
show that such explosions can replicate several of the peculiar properties of
\sn, including: the early UV/optical flash, a blue to red to blue color
transition, the moderately faint optical peak, red colors at maximum, and a
lack of IGE in the early spectra.

The appearance of double detonation SNe is effectively determined by two
properties: the mass of the C/O core and the mass of the He shell. The total
mass of the system determines the central density of the WD and thus the
amount of synthesized \radni. The \radni\ mass directly controls both the peak
luminosity and the kinetic energy of the explosion. High mass WDs ($\gtrsim
1.1\,M_\odot$) create enough \radni\ ($M_\mathrm{Ni} \gtrsim 0.5\,M_\odot$) to
produce large ($\gtrsim 1$4,000\,\kms) photospheric velocities and reach
normal brightness for a SN\,Ia, while low mass WDs ($\lesssim 0.9\,M_\odot$)
exhibit slower photospheric velocities ($\lesssim 1$0,000\,\kms) and produce
less \radni, therefore peaking at fainter luminosities \citep{Polin19}. That
we see both a high \ion{Si}{II} velocity and a low peak luminosity in \sn\
presents a challenge for the \citet{Polin19} double detonation models.
Furthermore, thick He shells ($M_\mathrm{He} \gtrsim 0.05\,M_\odot$) produce
more pronounced UV/optical flashes shortly after explosion, particularly in
conjunction with lower mass WDs, while thin He shells ($M_\mathrm{He} \lesssim
0.02\,M_\odot$) produce a more extreme color inversion in the days after
explosion.

We have attempted to model the evolution of \sn\ as a double detonation
explosion, following the procedure in \citet{Polin19}. We have specifically
focused on matching the photometric evolution (as noted above no models create
high-velocity ejecta and underluminous optical peaks), with particular
attention to the colors during the early flash and at maximum light. We find
that a model with $M_\mathrm{C/O} = 0.92\,M_\odot$ C/O core and a
$M_\mathrm{He} = 0.04\,M_\odot$ He shell best match \sn, as shown in
Figure~\ref{fig:double_det}.

\begin{figure*}
    \centering
    \includegraphics[width=\textwidth]{./figures/double_det.pdf}
    %
    \caption{Comparison of \sn\ to a dobule detonation model with a C/O core
    mass $M_\mathrm{C/O} = 0.92\,M_\odot$ and He shell mass $M_\mathrm{He} =
    0.04\,M_\odot$ (i.e., $M_\mathrm{WD} = 0.96\,M_\odot$). For the comparison
    we have adopted a model explosion time $t_\mathrm{exp} = t_\mathrm{fl} -
    0.72$\,d. \textit{Left}: Photometric comparison between \sn\ and the
    model. Symbols are the same as Figure~\ref{fig:p48}. The double detonation
    model provides a good match to the \rztf\ evolution, though the flux in
    the \gztf\ and \iztf\ bands is under- and over-predicted, respectively.
    The UV emission is also underestimated by the double detonation model.
    \textit{Right}: Spectroscopic comparison between \sn\ and the model.
    Observed spectra of \sn\ are shown in blue, with phases marked relative to
    \tbmax, whereas the model spectra are shown in dark grey, with phases
    marked relative to the modelled time of explosion. The modelled spectra
    have been smoothed with a Savitzky-Golay filter \citep{Savitzky64}. The
    photospheric velocity in the double detonation model is lower than what is
    observed in \sn, and the models feature more absorption and blanketing in
    the blue portion of the optical than what is observed. }
    %
    \label{fig:double_det}
\end{figure*}

While this model adequately matches the evolution of \sn\ in the \rztf\
filter, the predictions in the \gztf\ and \iztf\ bands do not match what is
observed. We show for the first time that there is an expected UV flash
associated with these double detonation models, however, our best fit model
underestimates the flux that was observed in the UV.

Synthesized spectra from our double detonation model exhibit features that are
not seen in \sn. The model spectra are dominated by \ion{Si}{II} absorption,
and show high-velocity absorption due to \ion{O}{I} and \ion{Ca}{II}, similar
to \sn. For our best-fit model, however, the \ion{Si}{II} velocities are too
slow, the \ion{Si}{II} $\lambda$5972 absorption is too strong, and the
\ion{S}{II} absorption too weak. Nuclear burning in the He shell creates heavy
elements in the outermost ejecta of double detonation explosions, leading to
deep \ion{Ti}{II} troughs and other blanketing in the blue-optical. Our model
exhibits a strong \ion{Ti}{II} absorption trough blueward of $\sim$4400\,\AA\
(see the $t_\mathrm{exp} + 9.25$\,d spectrum in Figure~\ref{fig:double_det}).
As was the case for models with extended clumps of \radni, the lack of such
absorption in \sn\ poses a challenge for the double detonation model.

With observations that probe a previously unexplored phase in the evolution of
such explosions, \sn\ provides an opportunity to determine where the double
detonation models must improve. It is possible that such improvements could
lead to better agreement with \sn. For instance, the nuclear reaction networks
and 1D models in \citet{Polin19} always burn the He shells to nuclear
statistical equilibrium. It is not unreasonable to think that 2D or 3D models,
with a more sophisticated nuclear reaction network, would create more IMEs and
less IGEs in the He shell, and that the ratio of the two created in the shell
could be highly dependent upon the line of sight. This could explain the lack
of IGEs and strong \ion{Si}{II} absorption seen in the early spectra, while
less IGEs in the outer layers would also reduce some of the line blanketing
seen around maximum light. This would lead to less reprocessing of blue
photons, perhaps creating better agreement between the models and photometry,
particularly in the \gztf\ band. The velocity discrepancy could also
potentially be explained as a line of sight effect. If the ignition of the WD
occurred off center, then the ejecta aligned with the cite of the initial He
ignition may receive a boost in velocity. The discrepancies in the UV are less
worrisome. While we show a qualitative UV flash, the magnitude of this flash
will be highly sensitive to the precise temperature and composition in the
very outer most ejecta, and thus any of the changes discussed above could
easily boost the model flux in the UV.

\subsection{Violent Mergers and Circumstellar
Interaction}\label{sec:merger_csm}

\citet{Piro16} show that circumstellar material in the vicinity of a WD at the
time of explosion can give rise to an early flash or bump in the SN\,Ia light
curve. Using a 1D toy model, with an assumed circumstellar density profile
$\propto r^{-3}$ and grey opacities, \citet{Piro16} find that the peak of the
early emission is roughly proportional to the extent of the circumstellar
material, while the duration of the flash is proportional to the square root
of the circumstellar mass. While the brightest model from \citet{Piro16} has a
flash brightness that peaks at $M_V \approx -15$\,mag, circumstellar material
that extends beyond $\sim$10$^{12}$\,cm could give rise to a flash that peaks
at $M_g \lesssim -16.4$\,mag, as is observed in \sn.

There are few proposed WD explosion models that produce dense circumstellar
material in the vicinity of the WD at the time of explosion, however. A
noteable exception is the violent merger, so called because the thermonuclear
explosion happens while the merger is still ongoing, of two C/O WDs
\citep{Pakmor10,Pakmor11,Pakmor12}. DD mergers should produce a wide variety
of circumstellar configurations depending on the initial parameters of the
inspiralling binary, which would, in turn, produce different signals shortly
after explosion (e.g., \citealt{Raskin13,Levanon19}).\footnote{Indeed, the
large number of potential configurations makes it very difficult to rule out
or select any specific circumstellar interaction scenario.}

Given the vast parameter space populated by different circumstellar
configurations, we are going to proceed under the (potentially poor)
assumption that such interaction could reproduce the UV/optical flash seen in
\sn. Following this assumption, a relevant question is -- can violent mergers
reproduce the properties of \sn\ in the days before and weeks after \tbmax?

In \citet{Kromer16}, the violent merger of two C/O WDs with masses of 0.9 and
0.76\,$M_\odot$ was simulated to recreate the properties of iPTF\,14atg, the
other SN\,Ia to show an early UV flash. A comparison of the low metallicity
model from \citet{Kromer16}, which was tuned for iPTF\,14atg, \textit{and not
\sn}, is shown in Figure~\ref{fig:violent_merger}.

\begin{figure*}
    \centering
    \includegraphics[width=\textwidth]{./figures/violent_merger.pdf}
    %
    \caption{Comparison of \sn\ to the low metallicity ($Z = 0.01 Z_\odot$)
    violent merger model of a 0.9\,$M_\odot$ WD and 0.76\,$M_\odot$ WD from
    \citet{Kromer16}. For the comparison we have adopted a model explosion
    time $t_\mathrm{exp} = t_\mathrm{fl} - 1.92$\,d. Thin lines represent one
    of 25 sightlines, while the bold lines represent a single sightline for
    illustrative purposes. \textit{Left}: Photometric comparison between \sn\
    and the model. Symbols are the same as Figure~\ref{fig:p48}. This
    particular violent merger model under-predicts the optical flux,
    especially in the \gztf\ and \rztf\ filters. The qualitative behavior,
    however, including red \gztf$ - $\rztf\ colors at peak and a lack of
    secondary maximum in the \iztf-band do match \sn. \textit{Right}:
    Spectroscopic comparison between \sn\ and the violent merger model.
    Observed spectra of \sn\ are shown in blue, with phases marked relative to
    \tbmax, whereas the model spectra are shown as thin grey lines. The thick
    black line highlights a specific viewing angle. The modelled spectra have
    been smoothed with a Savitzky-Golay filter \citep{Savitzky64}. The
    photospheric velocity in the violent merger model features lower
    velocities than \sn, while the strength of the the IME absorption is
    weaker in the models than what is observed.}
    %
    \label{fig:violent_merger}
\end{figure*}

The photometric evolution of this violent merger model qualitatively matches
\sn: (i) a moderately faint peak in the optical ($-17.6\,\mathrm{mag} \gtrsim
M_g \gtrsim -18.2$\,mag, depending on the viewing angle), (ii) red $g - r$
colors at peak, and (iii) a lack of a secondary maxmium in the $i$-band.
Furthermore, the spectra lack significant IGE absorption in the days after
explosion (right panel of Figure~\ref{fig:violent_merger}), as is observed in
\sn. A critical difference between \sn\ and violent merger models, is that the
merger models tend to produce relatively low expansion velocities (e.g.,
\citealt{Pakmor10,Kromer13a,Kromer16}). Indeed, this is one of the stark
differences between \sn\ and iPTF\,14atg, as iPTF\,14atg had a \ion{Si}{II}
$\lambda$6355 absorption velocity of $\sim$7,500\,\kms\ at peak, or roughly
half that observed in \sn. It is also clear from
Figure~\ref{fig:violent_merger} that the violent merger model from
\citet{Kromer16} exhibits weaker IME absorption than what is seen in \sn.

It is clear that additional modelling, likely of a different WD binary
configuration, is needed to better match \sn. For example, it is known that
higher mass WDs can produce a brighter peak \citep[e.g.,][]{Pakmor12}, which
would be more in line with \sn. It would also be beneficial to track the
unbound material following the DD merger, to see if the collision between this
material and the SN ejecta can replicate the early UV/optical flash seen in
\sn. If this feature can readily be recreated, it is possible that a violent
merger is responsible for \sn.

% \section{Rate of Thick He Shell Double Detonation Events}\label{sec:rates}
%
% \textbf{This may not actually be a thick shell}
%
% \aam{Use rough numbers from CLU and simple binomial calculation}

\section{Discussion and Conclusion}\label{sec:conclusions}

We have presented observations of the spectacular \sn, the second ever SN\,Ia
to exhibit a clear UV/optical flash in its early evolution. Despite this
dazzling, declarative display announcing \sn\ as a unique event among the
thousands of SNe\,Ia that have previously been cataloged, we find that \sn\
would be considered unusual even if the early flash had been missed.

The photometric evolution of \sn\ resembles that of the intermediate 86G-like
subclass of SNe\,Ia. With a moderately faint peak optical brightness ($M_g
\approx -18.5$\,mag), relatively fast decline [$\Delta m_{15}(g) = 1.3$\,mag],
and lack of a secondary maximum in the \iztf\ filter, \sn\ is clearly
distinguished photometrically from normal SNe\,Ia. These photometric
properties typically correspond to \citeauthor{Branch06}\ Cool SNe, yet the
spectroscopic evolution of \sn\ does match such events. \sn\ is a
\citeauthor{Branch06}\ Broad Line SN, with relatively weak \ion{Si}{II}
$\lambda$5972 absorption and large photospheric velocities. Furthermore, our
\texttt{TARDIS} models show little to no IGE present in the outer layers of
the SN ejecta, which further isolates \sn, even relative to other
\citeauthor{Branch06}\ Broad Line SNe. The fact that \sn\ exhibits
high-velocity \ion{Si}{II} $\lambda$6355 absorption and an underluminous peak
sets it apart from other SNe\,Ia.

We have found that building a consistent physical model to explain all of the
observed properties of \sn\ is challenging. Most models either replicate the
early flash but fail to reproduce the observed behavior around maximum light,
or vice versa.

We have examined four models in detail to try and explain the dramatic early
UV/optical peak in \sn, including: the collision of the SN ejecta with a
nondegenerate companion (e.g., \citealt{Kasen10a}), extended clumps of \radni\
in the outer layers of the SN ejecta (e.g., Magee \& Maguire 2020), the double
detonation explosion of a sub-Chandrasekhar mass WD \citep[e.g.,][]{Polin19},
and the violent merger of two sub-Chandrasekhar mass WDs
\citep[e.g.,][]{Kromer16}.

The SN ejecta-companion models, which can easily replicate the early UV flash
from \sn, simultaneously over-predict the optical flux at similar epochs.
Models with extended clumps of \radni\ produce significant blanketing in the
blue optical. We therefore favor either a double detonation explosion or a
(violent) merger of two WDs as the origin of \sn. These models are not without
their own shortcomings when it comes to matching the observations. While the
double detonation model produces an early flash and \rztf\ evolution that
provides a good match to \sn, it too produces blanketing that is too strong in
the blue optical and features photospheric velocities that are much lower than
what is observed. The specific WD merger model from \citet{Kromer16} that we
compare to \sn\ does a poor job of replicating the observations. Many of the
qualitative features match, however, so it is not unreasonable to think that
with some tuning (e.g., higher mass WDs) that the merger model could better
reflect what is observed in \sn.

Nebular spectra of \sn\ will play a crucial role in disambiguating between
these various scenarios. If the ejecta have collided with a nondegenerate
companion, then they will have stripped some surface material from the
companion, which will be revealed via narrow Balmer lines in the nebular phase
\citep[e.g.,][]{Wheeler75}. Alternatively, \citet{Polin19a} recently showed
that double detonation explosions with $M_\mathrm{WD} \lesssim 1.0\,M_\odot$
produce strong [\ion{Ca}{II}] $\lambda\lambda$7291, 7324 emission in the
nebular phase. Finally, violent mergers are expected to exhibit narrow
[\ion{O}{I}] $\lambda\lambda$6300, 6364 emission in their nebular spectra, as
unburned O from the disrupted WD is present at low velocities in the central
ejecta \citep{Kromer16}. Each of these predictions are unique to the scenarios
discussed here.

The critical challenge moving forward in understanding \sn-like events is the
rapid acquisition of \textit{Swift} UV observations shortly after explosion.
ZTF, and other similar surveys (ATLAS, ASAS-SN), have demonstrated the ability
to routinely find extremely young SNe\,Ia. Following this the challenge is to
(i) recognize these events as likely SNe\,Ia \textit{at the epoch of
discovery} (i.e., without a significant delay to obtain a spectroscopic
classifcation) and (ii) promptly obtain \textit{Swift} photometry. Only after
these UV observations become as routine as the discoveries themselves will we
be able to statistically answer questions about the nature of SN\,Ia
progenitors. \todo{as is always the case - the last paragraph I've written
sucks}


\acknowledgements

A.A.M.~is funded by the Large Synoptic Survey Telescope Corporation, the
Brinson Foundation, and the Moore Foundation in support of the LSSTC Data
Science Fellowship Program; he also receives support as a CIERA Fellow by the
CIERA Postdoctoral Fellowship Program (Center for Interdisciplinary
Exploration and Research in Astrophysics, Northwestern University).

This work is based on observations obtained with the Samuel Oschin Telescope
48-inch and the 60-inch Telescope at the Palomar Observatory as part of the
Zwicky Transient Facility project. ZTF is supported by the National Science
Foundation under Grant No. AST-1440341 and a collaboration including Caltech,
IPAC, the Weizmann Institute for Science, the Oskar Klein Center at Stockholm
University, the University of Maryland, the University of Washington,
Deutsches Elektronen-Synchrotron and Humboldt University, Los Alamos National
Laboratories, the TANGO Consortium of Taiwan, the University of Wisconsin at
Milwaukee, and Lawrence Berkeley National Laboratories. Operations are
conducted by COO, IPAC, and UW.

MMT Observatory access was supported by Northwestern University and the
Center for Interdisciplinary Exploration and Research in Astrophysics (CIERA).

This work made use of data supplied by the UK \textit{Swift} Science Data
Centre at the University of Leicester.

\software{\texttt{astropy} \citep{Astropy-Collaboration13}, 
          \texttt{corner} \citep{Foreman-Mackey16},
          \texttt{emcee} \citep{Foreman-Mackey13},
          \texttt{matplotlib} \citep{Hunter07}, 
          \texttt{pandas} \citep{McKinney10},
          \texttt{SALT2} \citep{Guy07},
          \texttt{scipy} \citep{2020SciPy-NMeth}, 
          \texttt{sncosmo} \citep{Barbary16},
          \texttt{SNooPY} \citep{Burns11},
          \texttt{TARDIS} \citep{Kerzendorf14},
          \texttt{TURTLS} \citep{Magee18}
          }

%% For this sample we use BibTeX plus aasjournals.bst to generate the
%% the bibliography. The sample63.bib file was populated from ADS. To
%% get the citations to show in the compiled file do the following:
%%
%% pdflatex sample63.tex
%% bibtext sample63
%% pdflatex sample63.tex
%% pdflatex sample63.tex

\bibliography{/Users/adamamiller/Documents/tex_stuff/papers}
\bibliographystyle{aasjournal}

%% This command is needed to show the entire author+affiliation list when
%% the collaboration and author truncation commands are used.  It has to
%% go at the end of the manuscript.
%\allauthors

%% Include this line if you are using the \added, \replaced, \deleted
%% commands to see a summary list of all changes at the end of the article.
%\listofchanges

\begin{deluxetable*}{ccccc}
\tabletypesize{\scriptsize}
\tablewidth{0pt}
\tablecaption{Log of Spectroscopic Observations for SN~2019yvg.\label{tab:spectra}}
\tablehead{
\colhead{Date (UT)}&
\colhead{MJD}&
\colhead{Phase$^{*}$}&
\colhead{Telescope+Instrument}&
\colhead{Range}\\
\colhead{}&
\colhead{(days)}&
\colhead{(days)}&
\colhead{}&
\colhead{(\AA)}}
\startdata
31 Dec 2019 &  58848.273298   &    &  LT+SPRAT$^{*}$ &  4000--9000?   \\
03 Jan 2020 &     &     &  LT+SPRAT$^{*}$ &  4000--9000   \\
04 Jan 2020 &     &     &  LT+SPRAT &  4000--9000   \\
12 Jan 2020 &     &     &  LT+SPRAT &  4000--9000   \\
12 Jan 2020 &     &     &  LT+SPRAT  &  4000--9000   \\
15 Jan 2020 &     &     &  P60+SEDM$^{**}$ &  4000--9000   \\
18 Jan 2020 &     &     &  P60+SEDM &  4000--9000   \\
24 Jan 2020 &     &     &  MMT+Binospec$^{***}$ &  4000--9000   \\
25 Jan 2020 &     &     &  Keck+LRIS$^{****}$ &  4000--9000   \\
27 Jan 2020 &     &     &  P60+SEDM  &  4000--9000   \\
29 Jan 2020 &     &     &  NOT+ALFOSC$^{******}$ &  4000--9000   \\
01 Feb 2020 &     &     &  P60+SEDM &  4000--9000   \\
\enddata
%%%%%%%%% needs update  there are 19 spec now 20200405 /jesper
\tablenotetext{*}{Liverpool Telescope}
%\tablenotetext{**}{}
%\tablenotetext{***}{}
\end{deluxetable*}



\end{document}