%% using aastex version 6.3
\documentclass[twocolumn]{aastex63}

\newcommand{\vdag}{(v)^\dagger}
\newcommand\aastex{AAS\TeX}
\newcommand\latex{La\TeX}

%%%%%%%%%%%%%%%%%%%%%%%%%%%%%%%%%%%%%%%%%%%%%%%%%%%%%%%%%%%%%%%%%%%%%%%%%%%%%%%%
%%
%% The following section defines new commands for comments from co-authors
%%
\definecolor{DarkOrange}{RGB}{204, 85, 0}
\definecolor{LincolnGreen}{RGB}{17, 102, 0}
\definecolor{Rust}{HTML}{9B4F0F}
\definecolor{DarkCyan}{HTML}{008B8B}
\definecolor{MediumAquaMarine}{HTML}{66CDAA}


\def\ion#1#2{#1$\;${\footnotesize\rm{#2}}\relax}

\newcommand{\kate}[1]{{\color{red} KM: \textbf{#1}}}
\newcommand{\steve}[1]{{\color{DarkCyan} Steve S: \textbf{#1}}}
\newcommand{\magee}[1]{{\color{Rust} MM: \textbf{#1}}}
\newcommand{\abi}[1]{{\color{LincolnGreen} AP: \textbf{#1}}}
\newcommand{\yy}[1]{{\color{blue} YY: \textbf{#1}}}
\newcommand{\aam}[1]{{\color{DarkOrange} aam: \textbf{#1}}}

\newcommand{\fromkate}[1]{{\color{brown} fromKM: {#1}}}
\newcommand{\frommark}[1]{{\color{orange} fromMM: {#1}}}
\newcommand{\frommb}[1]{{\color{purple} fromMB: {#1}}}
\newcommand{\fromabi}[1]{{\color{teal} fromAP: {#1}}}

\newcommand{\stockholm}[1]{{\color{cyan} stockholm: {#1}}}
\newcommand{\todo}[1]{{\color{magenta} to-do: {#1}}}

\newcommand{\rztf}{$r_\mathrm{ZTF}$}
\newcommand{\gztf}{$g_\mathrm{ZTF}$}
\newcommand{\iztf}{$i_\mathrm{ZTF}$}
\newcommand{\tfl}{$t_\mathrm{fl}$}
\newcommand{\trise}{$t_\mathrm{rise}$}
\newcommand{\tbmax}{$T_{B,\mathrm{max}}$}
\newcommand{\kms}{km\,s$^{-1}$}
\newcommand{\RSiII}{$\mathcal{R}($\ion{Si}{II}$)$}
\newcommand{\radni}{$^{56}$Ni}

\newcommand{\sn}{SN\,2019yvq}

%%
%%%%%%%%%%%%%%%%%%%%%%%%%%%%%%%%%%%%%%%%%%%%%%%%%%%%%%%%%%%%%%%%%%%%%%%%%%%%%%%%

%% Reintroduced the \received and \accepted commands from AASTeX v5.2
\received{\today}
\revised{}
\accepted{}
%% Command to document which AAS Journal the manuscript was submitted to.
%% Adds "Submitted to " the argument.
\submitjournal{ApJ}

%% For manuscript that include authors in collaborations, AASTeX v6.3
%% builds on the \collaboration command to allow greater freedom to 
%% keep the traditional author+affiliation information but only show
%% subsets. The \collaboration command now must appear AFTER the group
%% of authors in the collaboration and it takes TWO arguments. The last
%% is still the collaboration identifier. The text given in this
%% argument is what will be shown in the manuscript. The first argument
%% is the number of author above the \collaboration command to show with
%% the collaboration text. If there are authors that are not part of any
%% collaboration the \nocollaboration command is used. This command takes
%% one argument which is also the number of authors above to show. A
%% dashed line is shown to indicate no collaboration. This example manuscript
%% shows how these commands work to display specific set of authors 
%% on the front page.
%%
%% For manuscript without any need to use \collaboration the 
%% \AuthorCollaborationLimit command from v6.2 can still be used to 
%% show a subset of authors.
%
%\AuthorCollaborationLimit=2
%
%% will only show Schwarz & Muench on the front page of the manuscript
%% (assuming the \collaboration and \nocollaboration commands are
%% commented out).
%%
%% Note that all of the author will be shown in the published article.
%% This feature is meant to be used prior to acceptance to make the
%% front end of a long author article more manageable. Please do not use
%% this functionality for manuscripts with less than 20 authors. Conversely,
%% please do use this when the number of authors exceeds 40.
%%
%% Use \allauthors at the manuscript end to show the full author list.
%% This command should only be used with \AuthorCollaborationLimit is used.

%% The following command can be used to set the latex table counters.  It
%% is needed in this document because it uses a mix of latex tabular and
%% AASTeX deluxetables.  In general it should not be needed.
%\setcounter{table}{1}

%%%%%%%%%%%%%%%%%%%%%%%%%%%%%%%%%%%%%%%%%%%%%%%%%%%%%%%%%%%%%%%%%%%%%%%%%%%%%%%%
%%
%% The following section outlines numerous optional output that
%% can be displayed in the front matter or as running meta-data.
%%
%% If you wish, you may supply running head information, although
%% this information may be modified by the editorial offices.
\shorttitle{\sn\ is Fun and Cool}
\shortauthors{Miller et al.}
%%
%% You can add a light gray and diagonal water-mark to the first page 
%% with this command:
\watermark{DRAFT}
%% where "text", e.g. DRAFT, is the text to appear.  If the text is 
%% long you can control the water-mark size with:
%% \setwatermarkfontsize{dimension}
%% where dimension is any recognized LaTeX dimension, e.g. pt, in, etc.
%%
%%%%%%%%%%%%%%%%%%%%%%%%%%%%%%%%%%%%%%%%%%%%%%%%%%%%%%%%%%%%%%%%%%%%%%%%%%%%%%%%
\graphicspath{{./}{figures/}}
%% This is the end of the preamble.  Indicate the beginning of the
%% manuscript itself with \begin{document}.

\begin{document}

\title{\sn\  }

%% LaTeX will automatically break titles if they run longer than
%% one line. However, you may use \\ to force a line break if
%% you desire. In v6.3 you can include a footnote in the title.

%% A significant change from earlier AASTEX versions is in the structure for 
%% calling author and affiliations. The change was necessary to implement 
%% auto-indexing of affiliations which prior was a manual process that could 
%% easily be tedious in large author manuscripts.
%%
%% The \author command is the same as before except it now takes an optional
%% argument which is the 16 digit ORCID. The syntax is:
%% \author[xxxx-xxxx-xxxx-xxxx]{Author Name}
%%
%% This will hyperlink the author name to the author's ORCID page. Note that
%% during compilation, LaTeX will do some limited checking of the format of
%% the ID to make sure it is valid. If the "orcid-ID.png" image file is 
%% present or in the LaTeX pathway, the OrcID icon will appear next to
%% the authors name.
%%
%% Use \affiliation for affiliation information. The old \affil is now aliased
%% to \affiliation. AASTeX v6.3 will automatically index these in the header.
%% When a duplicate is found its index will be the same as its previous entry.
%%
%% Note that \altaffilmark and \altaffiltext have been removed and thus 
%% can not be used to document secondary affiliations. If they are used latex
%% will issue a specific error message and quit. Please use multiple 
%% \affiliation calls for to document more than one affiliation.
%%
%% The new \altaffiliation can be used to indicate some secondary information
%% such as fellowships. This command produces a non-numeric footnote that is
%% set away from the numeric \affiliation footnotes.  NOTE that if an
%% \altaffiliation command is used it must come BEFORE the \affiliation call,
%% right after the \author command, in order to place the footnotes in
%% the proper location.
%%
%% Use \email to set provide email addresses. Each \email will appear on its
%% own line so you can put multiple email address in one \email call. A new
%% \correspondingauthor command is available in V6.3 to identify the
%% corresponding author of the manuscript. It is the author's responsibility
%% to make sure this name is also in the author list.
%%
%% While authors can be grouped inside the same \author and \affiliation
%% commands it is better to have a single author for each. This allows for
%% one to exploit all the new benefits and should make book-keeping easier.
%%
%% If done correctly the peer review system will be able to
%% automatically put the author and affiliation information from the manuscript
%% and save the corresponding author the trouble of entering it by hand.

% \author[0000-0001-9515-478X]{A.~A.~Miller}
% \affiliation{Center for Interdisciplinary Exploration and Research in Astrophysics (CIERA) and Department of Physics and Astronomy, Northwestern University, 2145 Sheridan Road, Evanston, IL 60208, USA}
% \affiliation{The Adler Planetarium, Chicago, IL 60605, USA}
% \email{amiller@northwestern.edu}

\author{ZTF}

\author{et al.}

%% Note that the \and command from previous versions of AASTeX is now
%% depreciated in this version as it is no longer necessary. AASTeX 
%% automatically takes care of all commas and "and"s between authors names.

%% Mark off the abstract in the ``abstract'' environment. 
\begin{abstract}

\todo{Write the abstract}

\end{abstract}

%% Keywords should appear after the \end{abstract} command. 
%% See the online documentation for the full list of available subject
%% keywords and the rules for their use.
\keywords{}

\section{Introduction} \label{sec:intro}

\todo{write it; define and include references for ZTF}

\section{Discovery and Observations}\label{sec:obs}

\sn\ was discovered by K.~Itagaki, and detected at an unfiltered magnitude
of 16.7\,mag, in an image obtained on 2019 Dec 28.74 UT\footnote{UT times
are used throughout this paper}. The transient candidate was announced
$\sim$2\,hr later on the Transient Name Server (TNS), and given the
designation AT\,2019yvq \citep{Itagaki19}. Susequent spectroscopic
observations confirmed the SN nature of the transient, with an initial
report that the event was a SN Ib/c, and subsequent spectra confirming the
event as a SN Ia.\footnote{The initial classification is from
\citet{Kawabata20}, while the SN\,Ia classifications are from Prentice,
Mazzali, Teffs \& Medler and Dahiwale \& Fremling (see
\url{https://wis-tns.weizmann.ac.il/search?&name=SN2019yvq}).} These
spectroscopic observations also showed \sn\ to be at the same redshift as
NGC\,4441, its host galaxy.

\subsection{ZTF Photometric Observations}

ZTF simulataneously conducts multiple time-domain surveys using the ZTF
camera on the the Palomar Oschin Schmidt 48 inch (P48) telescope. \sn\ was
first detected by ZTF on 2019 Dec 29.46, as part of the ZTF public survey
(see \citealt{Bellm19a}). The automated ZTF pipeline \citep{Masci19}
automatically detected \sn, which passed internal thresholds (e.g.,
\citealt{Mahabal19}), leading to the production and disemination of a
real-time alert \citep{Patterson19}. The public alert included the position,
$\alpha = 12^{\mathrm{h}}27\arcmin21\farcs836$, $\delta =
+64\degr47\arcmin59\farcs87$ (J2000), and brightness, \rztf$ =
17.14\pm0.05$\,mag, which, together with the \citet{Itagaki19} discovery
report suggested the SN was fading. Continued monitoring with ZTF, and
follow-up with other telescopes, confirmed a spectacular decline in the early
emission from \sn\ (Figure~\ref{fig:p48}).

\begin{figure*}
    \centering
    \includegraphics[width=6in]{./figures/P48_lc.pdf}
    %
    \caption{Photometric evolution of \sn, highlighting the initial decline
    observed in the light curve. \gztf, \rztf, and \iztf\ observations are
    shown as filled green circles, open red circles, and filled golden
    crosses, respectively. UVOT $uvw1$ and $uvm2$ are shown as filled and
    open squares, respectively. The lower axis shows time measured in
    rest-frame days relative to the time of first light, \tfl\ (see
    \S\ref{sec:phot}), while the upper axis shows time relative to the time
    of $B$-band maximum, \tbmax. Note that the horizontal axis is shown with
    a linear scale from $0\,\mathrm{d} \le t - t_\mathrm{fl} \le 3$\,d and a
    log scale for $t - t_\mathrm{fl} > 3$\,d. Vertical grey ticks show
    epochs of spectroscopic observations.}
    %
    \label{fig:p48}
\end{figure*}

The field of \sn\ was additionally observed by ZTF with nearly a nightly
cadence as part of the ZTF partnership Uniform Depth Survey (ZUDS;
D.~Goldstein et al., in prep.). Using images obtained as part of the ZUDS
program, we perform forced PSF photometry at the location of \sn\ following
the procedure described in \citet{Yao19}.\footnote{Images obtained as part of
the ZTF public survey have not been released, preventing us from applying our
forced-PSF measurements. We therefore only include forced-PSF measurements in
the analysis described herein, though we note that our measurements are
largely consistent with those provided in the public alerts.} The evolution
of \sn\ in the \gztf, \rztf, and \iztf\ filters is shown in
Figure~\ref{fig:p48}.

\subsection{Other Photometric Observations}

Ultraviolet (UV) observations of \sn\ were obtained with the
Ultra-Violet/Optical Telescope (UVOT; \citealt{Roming05}) onboard the Neil
Gehrels Swift Observatory (hereafter \textit{Swift}; \citealt{Gehrels04})
following a time-of-opportunity request.\footnote{\textit{Swift} ToO
requests for \sn\ (\textit{Swift} Target ID: 13037) have been submitted by
D.~Hiramatsu, J.~Burke, and S.~Schulze.}
Pre-SN UVOT reference images are limited to the $uvw1$, $uvm2$, and $uvw2$
filters. As a result we cannot provide accurate estimates of the SN flux in
the \textit{Swift} $u$, $b$, and $v$ filters. We estimate the flux in the
$uvw1$, $uvm2$, and $uvw2$ filters using an aperture of size \steve{PPP}
pixels at the SN position, and subtract the flux measured using an identical
procedure in the pre-SN images \steve{Is this all correct?}. For clarity, we
only show the \textit{Swift} $uvw1$ and $uvm2$ light curves in
Figure~\ref{fig:p48}.\footnote{The $uvw2$ evolution is nearly identical to
$uvm2$. Furthermore, the red leak associated with the $uvw2$ filter (see
e.g., \citealt{Breeveld11}), in combination with the relatively red spectral
energy distribution of SNe\,Ia, make it very difficult to interpret $uvw2$
light curves of SNe\,Ia (see \citealt{Brown17} and references therein).
Therefore, unless otherwise noted, we exclude $uvw2$ measurements from the
analysis below.} \textit{Swift}/UVOT observations show that the initial
decline seen in the optical is even more dramatic in the UV.

While absolute flux measurements in the UVOT $u$, $b$, and $v$ filters are
not available, assuming the underlying flux from the host is constant, we
can estimate the time of $B$-band maximum, \tbmax, from the relative
$b$-band light curve. Using a second-order polynomial, we model the $b$-band
light curve near peak (including observations between JD$\,> \,$2,458,855.5
and JD$\,<\,$2,458,871.5). From this fit we estimate \tbmax$ =
$2,458,863.83$ \,\pm \,0.21$\,JD.

In parallel with the UV observations, \textit{Swift} observed \sn\ in the
X-rays with the X-ray Telescope \citep{Burrows05}. We do not detect variable
X-ray emission at the position of \sn, as is expected in SNe\,Ia (e.g.,
\citealt{Margutti12}). \todo{Note - if CSM interaction becomes the story of
this SN, then this paragraph needs to be revisited.}

\subsection{Optical Spectroscopy}

Spectroscopic observations of \sn\ were taken with a variety of telescopes
and instruments over multiple epochs beginning $\sim$2\,d after discovery
and continuing through $\sim$2\,months after \tbmax. An observing log is
listed in Table~\ref{tab:spectra}. The spectra were reduced using standard
procedures in \texttt{IDL}/\texttt{Python}/\texttt{Matlab}. The optical
spectral evolution of \sn\ is illustrated in Figure~\ref{fig:spec_evo}.

\begin{figure}
    \centering
    \includegraphics[width=3.35in]{./figures/spec_evo.pdf}
    %
    \caption{Observed spectral sequence of \sn. Spectra have been normalized
    by their median flux between 7200\,\AA\ and 7400\,\AA. The phase of each
    observation relative to \tbmax\ is shown to the right of the individual
    spectra. Prominent spectral features from intermediate mass elements are
    highlighted with vertical dashed lines. Some of the spectra show
    imperfect Telluric subtractions, giving rise to the non-smooth features
    around $\lambda_\mathrm{obs} \approx 7600$\,\AA.}
    %
    \label{fig:spec_evo}
\end{figure}




\section{NGC\,4441: the Host of \sn}\label{sec:host}

NGC\,4441 is the host galaxy of \sn. Sloan Digital Sky Survey (SDSS;
\citealt{York00}) spectroscopic measurements of the nucleus of NGC\,4441
yield a heliocentric-recession velocity of 2663\,\kms\ ($z_\mathrm{helio} =
0.00888$; \citealt{Abolfathi18}) and a \texttt{STARBURST} classification for
NGC\,4441. Morphologically, NGC\,4441 is classified as a peculiar,
weakly-barred, late-type lenticular galaxy (SABO$+$ pec;
\citealt{de-Vaucouleurs91}). SDSS images show a clear dust lane near the
center of the galaxy.

Using the 2M++ model of
\citet{Carrick15}, we estimate a peculiar velocity towards NGC\,4441 of
$+328.6$\,\kms, which combined with the recession velocity in the frame of
the cosmic microwave background\footnote{See
\url{https://ned.ipac.caltech.edu/velocity_calculator}} (CMB,
$v_\mathrm{CMB} = 2770.6$\,\kms), yields a total recession velocity $=
3099.2 \pm 150$\,\kms. The final uncertainty in the total recession velocity
is dominated by systematic uncertainties in the 2M++ model. We also note
that the 2M++ estimate is consistent, to within $\sim$5\%, with the Virgo
and Great Attractor infall models of \citet{Mould00}. Adopting $H_0 =
73$\,\kms\,Mpc$^{-1}$, we estimate the distance to NGC\,4441 to be $42.5 \pm
2.1$\,Mpc, corresponding to a distance modulus of $\mu = 33.14 \pm
0.11$\,mag, where the uncertainty on $\mu$ is dominated by the uncertainty
in the peculiar velocity correction. We hereafter adopt 33.14\,mag as the
distance modulus to NGC\,4441.\footnote{\citet{Tully13} estimate a
significantly smaller distance to NGC\,4441 ($\mu = 31.43 \pm 0.14$\,mag; $D
= 19.0$\,Mpc) based on surface brightness fluctuation measurements from
\citet{Tonry01}. If NGC\,4441 is at this distance, then \sn\ peaks at $M_g
\approx -16.8$\,mag, which is significantly underluminous for a SN\,Ia.
Given that \sn\ has a normal rise time $t_\mathrm{rise} \approx 18$\,d
(\S\ref{sec:phot}), relatively normal spectra at peak (\S\ref{sec:spec}),
and produced $\sim$0.4\,$M_\odot$ of \radni\ (\S\ref{sec:tardis}), it is
highly unlikely that it is nearly a factor of 10 less luminous in the \gztf\
filter than normal SNe\,Ia. We therefore adopt the larger kinematic distance
to NGC\,4441.}
% \todo{Lots of different catalogs provide
% metallicity for SDSS galaxies, should we report on these results at all?
% Port Z = 0.04 and Granada Z ~0.01, and do not agree, firefly ~ 0.6 0r 0.2
% (solar?) depending on weighting, probably not a great idea} \fromkate{that's
% a reasonable spread in values but likely different calibrations. I don't
% know which one is best or the references for comparing this value to. }

We estimate the total redenning towards \sn\ to be small. There is
relatively little line of sight extinction due to the Milky Way, $E(B-V)
\approx 0.018$\,mag \citep{Schlafly11, Schlegel98}. Furthermore, we do not
find significant evidence for strong extinction in NGC\,4441 \frommb{Maybe
worth noting how estimate below would change for peculiar dust as that seen
in some SNe (e.g. SN 2014J, Rv=1.4)}. Figure~\ref{fig:NaD} highlights the
\ion{Na}{I} D absorption in the spectrum of \sn\ due to gas in NGC\,4441 and
the Milky Way from our highest-resolution spectrum, $R \approx 4000$,
obtained with Binospec+MMT. The \ion{Na}{I} D absorption is weak, and we
estimate a total equivalent width (EW) $= 390$\,m\AA\ for NGC\,4441 and
$220$\,m\AA\ for the Milky Way. There is a systematic uncertainty of
$\sim$10\% on each of these estimates due to uncertainties in the
continuum-fitting procedure. Assuming similar properties for the dust in
NGC\,4441 and the Milky Way, we scale the color excess measurement for the
Milky Way by the ratio of \ion{Na}{I} D EWs to estimate $E(B-V) \approx
0.032$\,mag for \sn\ due to absorption in NGC\,4441. This yields a total
color excess towards \sn\ of $E(B-V) \approx 0.05$\,mag, which we adopt for
the subsequent analysis in this study. We note that this estimate is
consistent, to within the uncertainties, with the EW(\ion{Na}{I}
D)--$E(B-V)$ relations presented in \citet{Poznanski12}. Further supporting
the claim of low extinction is the lack of a detection of the \ion{K}{I}
$\lambda\lambda$7665, 7699 interstellar lines or the diffuse interstellar
band at 5780\,\AA, which also serve as proxies for extinction (e.g.,
\citealt{Phillips13}).

\begin{figure}
    \centering
    \includegraphics[width=3.35in]{./figures/NaD.pdf}
    %
    \caption{Zoom-in on our moderate resolution ($R \approx 4000$)
    MMT+Binospec spectrum of \sn\ highlighting absoprtion due to \ion{Na}{I}
    D in the host galaxy, NGC\,4441 (blue solid line), and the Milky Way
    (thin black line). The velocity scale is centered on the D$_1$ line in
    NGC\,4441, with the SDSS redshift shown via the vertical dashed line. No
    shift has been applied to the Milky Way lines. The \ion{Na}{I} D lines,
    which serve as a proxy for dust-obscuration along the line of sight
    (e.g., \citealt{Poznanski12,Phillips13}) are weak, indicating a
    relatively small amount of reddening.}
    %
    \label{fig:NaD}
\end{figure}



\section{Photometric Analysis}\label{sec:phot}

\subsection{The Time of First Light, \tfl}\label{sec:t_fl}

We estimate the time of first light, \tfl, for \sn\ following the procedure
described in \citet{Miller20}. Briefly, \citet{Miller20} model the early
emission from a SN Ia as a power-law in time, $f \propto (t -
t_\mathrm{fl})^\alpha$, where $f$ is the flux, $t$ is time, and $\alpha$ is
the power-law index. \tfl\ is assumed to be the same everywhere in the
optical, allowing us to simultaneously fit observations in each of the ZTF
filters.

An important caveat for \sn\ is that the observed early decline in the light
curve clearly does not follow the simple power-law model, and thus these
observations must be masked when performing the fit. We conservatively
exclude observations from the first two nights of ZTF detection from the fit
(this choice is conservative as it is unclear when the mechanism that powers
the initial bump in \sn\ no longer significantly contributes to the flux in
the \gztf\ and \rztf\ filters). From the fit we find \tfl$ = -17.5
\pm^{1.0}_{1.3}$\,d relative to \tbmax. We know that the time of explosion
must be $< -17.4$\,d based on the discovery detection in \citealt{Itagaki19},
and, by definition $t_\mathrm{fl} \ge t_\mathrm{exp}$, meaning a portion of
the posterior distribution for our model cannot be correct. We also find
$\alpha_g = 2.15 \pm^{0.49}_{0.36}$ and $\alpha_r = 1.91 \pm^{0.42}_{0.31}$.
These values are typical of the normal SNe Ia studied in \citet{Miller20}. If
we only exclude the first observation from the model fit we find
significantly different results with a rise time that increases by $\sim$5\,d
and power-law indicies that increase by $\gtrsim 1$.

\subsection{Luminosity of the Initial UV/optical Flash}

\todo{discussion of luminosity, teff, etc}

\subsection{Maximum Light and Decline}

While the rise time and power-law indicies of \sn\ are similar to other
normal SNe Ia, the photometric evolution does not resemble normal SNe Ia.
\sn\ is underluminous ($M_{g,\mathrm{max}} \approx -18.5$\,mag), declines
rapidly ($\Delta m_{15}(g) \approx 1.4$\,mag) \todo{measure precisely and
provide uncertainties}, and does not exhibit a ``shoulder'' in the \rztf\ or a
secondary maximum in the \iztf\ light curves post-maximum \fromkate{A plot of the light curves comparing to the Yao sample would be good to compare brightness, secondary max and deltam15}. For context, of
the 127 SNe found by ZTF and studied in \citet{Yao19}, only 1, ZTF18abclfee
(SN\,2018crl), a SN\,2002cx-like event, had a faster decline than \sn. The
underluminous, fast-declining evolution of \sn\ is broadly consistent with
the SN\,1991bg-like subclass of SNe Ia (e.g., \citealt{Taubenberger17}),
however, we show that \sn\ is spectroscopically distinct relative to
91bg-like SNe (\S\ref{sec:spec}). We also find that standard SN Ia fitting
techniques, including \texttt{SALT2} \citep{Guy07} and \texttt{SNooPY}
\citep{Burns11}, do not provide good matches to the evolution of \sn. \fromkate{Explain briefly why they don't work.}

\subsection{Color Evolution}

\sn\ is further distinguished from normal SNe Ia by its unusual color
evolution (Figure~\ref{fig:colors}). The top panel of Figure~\ref{fig:colors}
shows the \gztf$ - $\rztf\ evolution of 35 normal SNe Ia with ZTF
observations within 3\,d of \tfl\ (see \citealt{Bulla20}), with the color
evolution of \sn\ over-plotted. \sn\ exhibits a prominent blue to red to
blue, or ``red bump,'' evolution in the $\sim$week after \tfl. Similar red
bumps are only seen in $\sim$10\% of the ZTF sample \citep{Bulla20}.
Furthermore, while normal SNe Ia exhibit a large scatter in \gztf$ - $\rztf\
shortly after \tfl\ they evolve to form a tight locus around \tbmax. \sn\ is redder at peak than each of the normal SNe Ia in the \citet{Bulla20} sample,
and it exhibits a far more rapid decline in \gztf$ - $\rztf. In the
$\sim$20\,d after \tbmax, the \gztf$ - $\rztf\ color of \sn\ increases by
$\sim$1.1\,mag, whereas the typical normal SN Ia color only becomes
$\sim$0.5\,mag redder over the same time frame. 

\begin{figure}
    \centering
    \includegraphics[width=3.35in]{./figures/P48_colors.pdf}
    %
    \caption{Photometric color evolution of \sn. \textit{Top}: \gztf$ -
    $\rztf\ evolution of \sn\ (solid green squares), corrected for the total
    line of sight extinction (see \S\ref{sec:host}), and compared with the
    evolution of 35 normal SNe Ia (open circles) observed within 3\,d of
    \tfl\ by ZTF (from \citealt{Bulla20}). \sn\ is the reddest SN in the
    group, and it exhibits the fastest evolution to red colors post-\tbmax.
    \textit{Bottom}: the $uvm2 - uvw1$ (purple crosses), \gztf$ - $\rztf\
    (solid, green squares), and \rztf$ - $\iztf\ (open, red squares) color
    evolution of \sn. The ``red bump'' can clearly be seen in the
    optical.}
    %
    \label{fig:colors}
\end{figure}

The offset in the \gztf$ - $\rztf\ color evolution of \sn\ relative to normal
SNe\,Ia would be reduced if the reddenning towards \sn\ has been
significantly underestimated. A color excess of $E(B-V) \approx 0.25$\,mag,
rather than the 0.05\,mag adopted in \S\ref{sec:host}, would roughly align
the pre-\tbmax\ \gztf$ - $\rztf\ color of \sn\ with the tight locus seen in
Figure~\ref{fig:colors}. Such a correction would also bring the peak optical
brightness of \sn\ in line with normal SNe\,Ia (for $E(B-V) \approx
0.25$\,mag, $M_g \approx -19.25$\,mag and $M_r \approx -19.1$\,mag for \sn).
Such a large reddening would dramatically change the appearance of \sn\ in
the UV. The bottom panel of Figure~\ref{fig:colors} shows the $uvm2 - uvw1$
color evolution of \sn. At \tbmax, $uvm2 - uvw1 \approx 1.2$\,mag, making
\sn\ one of the most UV blue SNe\,Ia at maximum light (e.g.,
\citealt{Milne10,Brown17}). A color excess of $E(B-V)
\approx 0.25$\,mag is equivalent to $E(uvm2 - uvw1)
\approx 0.5$\,mag using the Fitzpatrick+ (1999) extinction law, $R_V = 3.1$, and the effective wavelengths of the \textit{Swift} filters from \citet{Brown17} \frommb{I get 0.65 mag, maybe double-check?}. This results in a $uvm2 - uvw1 \approx 0.XXX$\,mag at \tbmax,
making \sn\ the most UV blue SN\,Ia observed by \textit{Swift} \fromkate{this needs to be updated because none of the objects in Brown sample are corrected for extinction}. Furthermore,
this much extinction would dramatically increase the luminosity of the initial
peak in emission by a factor of $\sim$5. We conservatively estimate the
luminosity $\sim$1.2\,d after \tfl\ to be $\sim$2$\times
10^{42}$\,erg\,s$^{-1}$. It is difficult to explain such a large luminosity
(see below), increasing this by a factor of 5 and bringing it inline with the
typical peak luminosity of SNe\,Ia, would be nearly impossible to explain \fromkate{luminosity in which band? what would be nearly impossible to explain?} \frommb{The factor of 5 is likely for UV filters, while optical should have a factor of $\sim$2 for ebv=0.25 mag. So if you are referring to total luminosity, the increase would be smaller than 5. Also, note that some models from e.g. Polin predict bumps to be brighter than peak (although these thick helium models are likely ruled out anyway)}. We
therefore conclude that the color excess towards \sn\ is not underestimated,
and that the SN is instead intrinsically red in the optical.

Even if one ignores the striking initial bump in the light curve of \sn, we
can still conclude that \sn\ is not a normal SN Ia based on its other
photometric properties (e.g., low luminosity, fast decline, lack of a
near-infrared secondary maximum, red appearance at peak, and rapid evolution
to red colors after \tbmax).


\section{Spectral Evolution of \sn}\label{sec:spec}

%%%%%%%%%%%%%%%%%%%%%%%%%%%%%%%%%%%%%%%%%%%%%%%%%%%%%%%%%%%%%%%%%%%%%%%%%%%%%%%%%%%%%%%%%%%%%%%%%%%%%%%
%%%%%%%%%%%%%From KM/MM trying to bring together modelling and spectral analysis %%%%%%%%%%%%%%%%%%%%%%

Optical spectra of \sn\ were obtained at phases from $-$14.9\,d (2.6\,d after
\tfl) to 66.5\,d after \tbmax. Details of the spectra are presented in
Table~\ref{} and the spectral evolution is shown in Figure~\ref{fig:spec_evo}.
The absorption features in \sn\ are typical of SNe\,Ia, including intermediate
mass elements (IMEs), primarily Si, Ca, and O, as well as iron-group elements
(IGEs). 

\subsection{TARDIS Models}\label{sec:tardis}

To determine the structure of the ejecta and relative contributions of
different ions at early and maximum light phases, we have modelled the spectra
at $-$14.9\,d, $-12.0$\,d, and $+$0.0\,d using the 1D Monte Carlo radiative
transfer code \texttt{TARDIS} \citep{Kerzendorf14}. Parameters of our TARDIS
models are given in Table~\ref{tab:tardis}.

%%%Description of overall modelling results, velocities, features present, temperature. Then detailed discussion of Mg II vs Si III can be removed.
The first spectrum of \sn\ at $-$14.9\,d (2.6\,d after \tfl) shows shallow
features consistent with IMEs moving at extremely high velocities
($>$\,20,000\,\kms, Figure~\ref{fig:spec_evo}). The best-fitting
\texttt{TARDIS} model is shown in Figure~\ref{fig:tardis}. Our model
demonstrates that the shallow absorption features observed at this phase can
be reproduced solely by IMEs (predominantly \ion{Si}{II}), and that the
presence of IGE is not required to match the data. In Figure~\ref{fig:tardis},
we show the contribution of individual elements to the spectra. Our model also
confirms the high velocities of the ejecta -- we find the spectral features
and temperature are best reproduced with a photospheric velocity of
$\sim$25,000\,\kms. Compared to modelling of the spectroscopically similar
SN\,2002bo \citet{Stehle05}, see below, we find \sn\ has a lower photospheric
temperature ($\sim$8,000\,K, compared to $\sim$9,500\,K for SN\,2002bo).

\begin{figure*}
    \centering
    \includegraphics[width=6.0in]{./figures/good_model_pl_10.pdf}
    %
    \caption{PRELIMINARY PRELIMINARY PRELIMINARY}
    %
    \label{fig:tardis}
\end{figure*}

Similarly, for the $-12.0$\,d spectrum we find that a model that does not
contain IGE above $\sim$16,500\,\kms\ reproduces the majority of the
spectroscopic features. At this phase the model suggests the photospheric
temperature has not significantly changed.

\mark{Is there more to say about TARDIS models at maximum light?}

\subsection{\ion{Si}{II} Evolution}\label{sec:SiII}

We have measured the velocity of the \ion{Si}{II} $\lambda$6355 absorption
feature following the procedure described in \S2.5 of \citet{Maguire14}. We
have also estimated the pseudo-equivalent widths (pEWs) of the \ion{Si}{II}
$\lambda\lambda$5972, 6355 features, allowing us to measure their ratio, also
known as the $\mathcal{R}($\ion{Si}{II}$)$; see \citet{Hachinger08} for the
updated definition relative to \citet{Nugent95}.

\begin{figure}
    \centering
    \includegraphics[width=3.35in]{./figures/vel_evolution.pdf}
    %
    \caption{Velocity evolution of \ion{Si}{II} $\lambda$6355\,\AA\ absorption
    in \sn\ (large, filled circles). For comparison we also show the
    measurements for 264 SNe Ia observed by the Palomar Transient Factory
    (PTF) as open circles, with SN\,2010jn (PTF\,10ygu), the SN with the
    fastest moving ejecta in the PTF sample, highlighted via orange crosses.
    We additionally show the velocity evolution of SN\,2002bo, a SN that is
    very similar to \sn, as open diamonds. The median velocity evolution of
    each of the spectroscopic classes defined by \citet{Branch06} (Shallow
    Silicon, Core Normal, Broad Line, and Cool) are shown via solid lines. It
    is clear that \sn\ has exceptionally fast moving ejecta relative to
    typical SNe Ia.}
    %
    \label{fig:vel_evo}
\end{figure}

%%%Velocity and equivalent width measurements?
The velocity evolution of \ion{Si}{II} $\lambda$6355 is shown in
Figure~\ref{fig:vel_evo}, compared to measurements for the Palomar Transient
Factory (PTF) SN\,Ia sample from \citet{Maguire14} and the median velocity
evolution of SNe\,Ia belonging to the four different classes (Shallow Silicon,
Core Normal, Broad Line, and Cool) identified in
\citet{Branch06};\footnote{The velocity measurements are from
\citet{Blondin12}, while the method to determine the median velocity is
described in \citet{Miller18}.} hereafter, the \citeauthor{Branch06} class. 

At \tbmax, the pEW measurements for the \ion{Si}{II} $\lambda$6355 and
$\lambda$5972 features are $183\pm1$\,\AA, and $13\pm2$\,\AA, respectively,
unambiguously classifying \sn\ as a \citeauthor{Branch06} Broad Line SN\,Ia.
\sn\ stands out in Figure~\ref{fig:vel_evo} with some of the highest
\ion{Si}{II} velocities that have ever been observed. Within the PTF sample,
only SN\,2010jn (PTF10ygu) exhibits a \ion{Si}{II} absorption velocity as high
as \sn\ at every phase in its evolution.

As first noted by \citet{Nugent95}, and later confirmed by
\citet{Hachinger08}, $\mathcal{R}($\ion{Si}{II}$)$ is a luminosity indicator,
with larger values of $\mathcal{R}($\ion{Si}{II}$)$ corresponding to lower
luminosities. This correlation is driven by the ionization balance of
\ion{Si}{II}/\ion{Si}{III}, with cooler objects having stronger \ion{Si}{II}
$\lambda$5972 features. In Figure~\ref{fig:r_evo}, we show the
$\mathcal{R}($\ion{Si}{II}$)$ of \sn\ as a function of time, compared to
SN\,2011fe, SN\,2002bo and the PTF spectral sample of \citet{Maguire14}. \sn\
has a similar $\mathcal{R}($\ion{Si}{II}$)$ evolution to SN\,2002bo from high
to low values from early times up to maximum light. This evolution is
different to that of SN\,2011fe and the majority of the PTF SN\,Ia sample.

\begin{figure}
    \centering
    \includegraphics[width=3.35in]{./figures/R_evolution.pdf}
    %
    \caption{Evolution of the ratio of the pEW of \ion{Si}{II}
    $\lambda$5972\,\AA\ to \ion{Si}{II} $\lambda$6355\,\AA,
    $\mathcal{R}($\ion{Si}{II}$)$ in \sn\ (large, filled circles). For
    comparison we also show the measurements for 264 SNe Ia observed by the
    Palomar Transient Factory (PTF) as open circles. The velocity evolution
    of SN\,2002bo and SN\,2011fe are highlighted as open diamonds and open
    squares, respectively. \sn\ and SN\,2002bo exhibit an unusual inversion
    in $\mathcal{R}($\ion{Si}{II}$)$ as they evolve toward maximum light.}
    %
    \label{fig:r_evo}
\end{figure}

At face value, the $\mathcal{R}($\ion{Si}{II}$)$ evolution in
Figure~\ref{fig:r_evo} suggests that the effective temperature of \sn\
increases significantly as it rises to maximum light. The optical colors (see
Figure~\ref{fig:colors}), however, are nearly constant throughout this phase
while the UV$ - $optical colors clearly decline during the same period,
suggesting a decline in the effective temperature. The \texttt{TARDIS}
modelling also does not require a significant increase in temperature towards
maximum light from the temperature of $\sim$8,000\,K required at early times.
This stands in contrast to SN\,200bo, which increases in temperature from
$\sim$9,500\,K at $-12.9$\,d to $\sim$14,000\,K at maximum light
\citep{Stehle05}. This temperature is similar to \citeauthor{Branch06} Core
Normal SNe, such as SN\,2011fe, which typically have temperatures of
$\sim$14,500--15,000\,K at maximum light \citep{Mazzali14}.

\citet{Benetti04} interpreted these competing effects as the result of
significant \ion{Si}{II} mixing in the ejecta of SN\,2002bo. Mixing or Si
production in the outermost layers of the ejecta would (i) lead to larger
\ion{Si}{II} velocities, (ii) produce \ion{Si}{II} line ratios that indicate
cool temperatures (because there is less radioactive material to heat the
ejecta), before eventually (iii) producing low values of
$\mathcal{R}($\ion{Si}{II}$)$ as the photosphere recedes through the ejecta to
higher temperature regions. This picture is consistent with the
\citet{Stehle05} models of SN\,2002bo. In those models, Si completely
dominates the species at velocities above $\sim$23,000\,\kms, while there is
very little ($\sim$1\%) IGE above a 1.35\,$M_\odot$ in radial mass
coordinates. The same explanation does not clearly apply to \sn, however, as
our \texttt{TARDIS} models do not show evidence for a significant increase in
the photospheric temperature at maximum light. Our \texttt{TARDIS} models are,
unlike SN\,2002bo, consistent with no IGEs in the outer layers of the \sn\
ejecta. A possible explanation for the evolution of
$\mathcal{R}($\ion{Si}{II}$)$ in \sn\ is that as the photosphere receeds to
regions with more \radni, increased Si ionization occurs leading to a relative
decrease in \ion{Si}{II} $\lambda$5972 absorption despite relatively small
changes in the photospheric temperature.

\kate{and} \magee{This still seems like an unexplained puzzle to me...}

\subsection{Spectral Comparison}\label{sec:spec_comp}

%%%Comparison to other objects & Si II ratio
In Figure~\ref{fig:spec_comp}, we compare the spectral evolution of \sn\ to
two Broad-Line SNe, SN\,2002bo and SNe\,2010jn, and two Cool SNe, SN\,1986G
and SN\,2004eo \citep{Cristiani92,
Benetti04,Pastorello07,Silverman11,Hachinger13,Maguire14} at four phases,
pre-maximum, maximum, $\sim$1 week post maximum, and $\sim$6 weeks post
maximum. The evolution of \sn\ and SN\,2002bo is remarkably similar at all
phases. The only significant difference between the two is the absorption
trough at $\sim$4800\,\AA\ in the pre- and maximum-light spectra. This
feature, which is typically attributed to a combination of \ion{Fe}{II},
\ion{Fe}{III}, and \ion{Si}{II}, is extremely narrow in \sn. This is in
agreement with the \texttt{TARDIS} modelling results where no Fe is required
in the outer ejecta of \sn\ to match the observed spectra at early times.
SN\,2010jn, which exhibits large \ion{Si}{II} velocities like \sn, shows
weaker IME absorption and stronger IGE absorption than \sn. While the
\citeauthor{Branch06}Cool SNe\,1986G and 2004eo feature lower velocities than
\sn, there is strong agreement in the relative \ion{Si}{II} line strengths of
SN\,1986G and the earliest spectra of \sn. \todo{86G had E(B-V) $\sim$ 0.9,
correct spectra for that?} \fromkate{yes.}

\begin{figure*}
    \centering
    \includegraphics[width=7.25in]{./figures/spec_comp.pdf}
    %
    \caption{Spectral comparison of \sn\ to \citet{Branch06} Broad Line and
    Cool SNe\,Ia. \textit{Left panel}: pre-maximum spectra showing the
    similarity of \sn\ and SN\,2002bo. While the expansion velocities in the
    Cool SN\,1986G spectrum are considerably lower than those in the Broad
    Line SNe, the relative ratios of the \ion{Si}{II} features are similar.
    \textit{Second panel}: Comparison of \sn\ to the Broad Line SNe\,2002bo
    and SN\,2010jn. These SNe all feature nearly identical maximum-light
    spectra. By this phase, the relative strength of the \ion{Si}{II}
    absorption features is no longer similar to \citet{Branch06} Cool SNe, as
    illustrated by SN\,2004eo. \textit{Third panel}: $\sim$1 week post-maximum
    spectra. \textit{Fourth panel}: Transitional phase spectra. Comparison
    spectra have been downloaded from WISeREP \citep{Yaron12}, with spectra
    for individual SNe from the following sources: SN\,1986G --
    \citet{Cristiani92}, SN\,2002bo -- \citet{Benetti04,Silverman11},
    SN\,2010jn (PTF\,10ygu) -- \citet{Hachinger13,Maguire14}, SN\,2004eo --
    \citet{Pastorello07}. \todo{Should this be 2 figures?}}
    %
    \label{fig:spec_comp}
\end{figure*}

The maximum-light spectra shown in the second panel of
Figure~\ref{fig:spec_comp} reveal a much higher luminosity/temperature for
\sn, as the \ion{Si}{II} $\lambda$5972\,\AA\ absorption has nearly disappeared
around \tbmax\ (see discussion of $\mathcal{R}($\ion{Si}{II}$)$ in
\S\ref{sec:SiII}). The appearance of \sn, SN\,2002bo, and SN\,2010jn are all
similar at this epoch, with the exception of the 4800\,\AA\ feature mentioned
above. SN\,2004eo has a similar appearance to \sn, though it has lower
velocities and cooler temperatures (as traced by \ion{Si}{II} $\lambda$5972).

The $+9.2$\,d spectrum of \sn, shown in the third panel of
Figure~\ref{fig:spec_comp}, shows absorption due to IGE. Additional
differences between \sn\ and SN\,2002bo can be seen at this phase. There is
stronger absorption in \sn\ blueward of \ion{Ca}{II} H\&K, and the \ion{S}{II}
``W'' absorption feature is still present in \sn\ and it cannot be identified
in SN\,2002bo or SN\,2010jn. SN\,2004eo maintains an appearance that is
somewhat similar to \sn, though as before, the temperatures are cooler and the
velocities lower.

Spectra obtained $\sim$6 weeks after maximum light are shown in the fourth
panel of Figure~\ref{fig:spec_comp}. By this time, as the SNe are
transitioning into a nebular phase, the appearance of each spectrum is similar
modulo some minor differences in the relative line strengths of different
features.

\frommb{I wonder if somewhere in Section 3 we should point out that
low-luminosity and high velocities are unprecedented (see data in Figure 11 of
Polin et al. 2019)...} \fromabi{I agree. This is unlike any of the data point
I could get my hands on and I think this is worth stating, because it is hard
to come up with something that has little Ni (low luminosity) but has enough
kinetic energy to boost to these velocities.}

%%%%%%%%%%%%END KM/MM rewrite %%%%%%%%%%%%%%%%%%%%%%%%%%%%%%%%%%%%%%%%%%%%%%%%%%%%%%%%%%%%%%%%%%%%%%%%%
%%%%%%%%%%%%%%%%%%%%%%%%%%%%%%%%%%%%%%%%%%%%%%%%%%%%%%%%%%%%%%%%%%%%%%%%%%%%%%%%%%%%%%%%%%%%%%%%%%%%%%%









% Our initial spectra of \sn\ are dominated by intermediate mass elements
% (IMEs), primarily Si, Ca, and O, moving at extremely high velocities ($>
% 20,000$\,\kms), as identified in Figure~\ref{fig:spec_evo}. In the spectra
% obtained $>$10\,d prior to \tbmax, there is also a strong absorption feature
% around 4200\,\AA\ that we tentatively identify as \ion{Mg}{II} $\lambda$ 4481. \fromkate{Mark on spectral plot with `?'}\frommark{Also supported by TARDIS modelling, which does not include Ti}.
% Though it is also possible that this feature corresponds to the Ti trough
% described in \citet{Polin19} for double-detonation SNe Ia. \todo{Could this
% feature be \ion{Si}{III}? could explain small notch at ~4500 Ang} \frommark{For both the 31/12 and 03/01 spectra, there are no strong indications of Si III from the TARDIS modelling. The temperature throughout is less than 8500 K in both spectra, which is about 1000 K less than 02bo (Stehle+05). As noted by Adam, if there is an underlying contribution from e.g. interaction, this would affect our ability to model with TARDIS and deserves mentioning} While the
% early spectra are dominated by IMEs, we find no evidence for \ion{C}{II}
% absorption in the spectra of \sn. \fromkate{I think the TARDIS modelling should be introduced early on in this section so that the speculation is more grounded and based on the modelling results}
%
% Another interesting feature of the early spectra is the relative prominence of
% \ion{Si}{II} $\lambda$5972 relative to \ion{Si}{II} $\lambda$6355. Prominent
% \ion{Si}{II} $\lambda$5972 absorption suggests a relatively cool photosphere,
% however, by the time of maximum light this feature has almost disappeared
% entirely. In our maximum-light spectrum of \sn, we estimate pseudo-equivalent
% widths (pEWs) of $183\pm1$\,\AA, and $13\pm2$\,\AA\ for \ion{Si}{II}
% $\lambda$6355 and $\lambda$5972, respectively. These pEW measurements
% unambiguously classify \sn\ as a broad-line SN Ia according to the
% classification scheme presented in \citet{Branch06}. The high velocities
% typically associated with this class also match the early evolution of \sn.



% The velocity evolution of the \ion{Si}{II} $\lambda$6355 absorption feature
% in \sn\ is shown in Figure~\ref{fig:vel_evo}. These measurements have been
% made following the procedure described in \citet{Maguire14} (see their
% \S2.5). For comparison, we also show the \citet{Maguire14} measurements of
% the \ion{Si}{II} velocity evolution of 264 SNe Ia observed by the Palomar
% Transient Factory (PTF) as open circles in Figure~\ref{fig:vel_evo}, as well
% as the median velocity evolution of SNe belonging to the four different
% classes (Shallow Silicon, Core Normal, Broad Line, and Cool) identified in
% \citet{Branch06}.\footnote{The velocity measurements are from
% \citet{Blondin12}, while the method to determine the median velocity is
% described in \citet{Miller18}.} \sn\ stands out among other SNe Ia with
% some of the highest \ion{Si}{II} velocities that have ever been observed.
% Within the PTF sample, only SN\,2010jn exhibits a \ion{Si}{II} absorption
% velocity as high as \sn\ at every phase in its evolution. \todo{discuss other
% BL with fast Si: sn1997bq , sn2002cd, sn2003W?} \fromkate{I wouldn't worry too much about the discussion of these objects since they're now old}

% \citet{Benetti04} argue that \ion{S}{II} $\lambda$5640\,\AA, as a weak line,
% is more likely to be formed close to the continuum photosphere, and therefore
% serves as a better tracer of the photospheric velocity than \ion{Si}{II}
% $\lambda$6355\,\AA. We measured the absorption velocity of this line using
% the same method described above, and find that the \ion{S}{II} absorption
% velocity is remarkably $\sim$3000\,\kms\ slower than \ion{Si}{II}
% $\lambda$6355\,\AA\ at all epochs where both features are measurable (from
% -12\,d to +0\,d relative to \tbmax). \todo{cut this paragraph?}
%
% In Figure~\ref{fig:r_evo} we show the ratio of the pEW for \ion{Si}{II}
% $\lambda$5972\,\AA\ to \ion{Si}{II} $\lambda$6355\,\AA\ [$\equiv
% \mathcal{R}($\ion{Si}{II}$)$; see \citealt{Hachinger08} for the updated
% definition relative to \citealt{Nugent95}] as a function of time. As first
% noted by \citet{Nugent95}, and later confirmed by \citet{Hachinger08},
% $\mathcal{R}($\ion{Si}{II}$)$ is a luminosity indicator, with larger values of
% $\mathcal{R}($\ion{Si}{II}$)$ corresponding to lower luminosities.

% Shortly after explosion, \sn\ exhibits large values of
% $\mathcal{R}($\ion{Si}{II}$)$, yet, by maximum light
% $\mathcal{R}($\ion{Si}{II}$)$ declines to low values. Compared to the PTF
% spectroscopic sample (\citealt{Maguire14}; open circles in
% Figure~\ref{fig:r_evo}), \sn\ has one of the largest values of
% $\mathcal{R}($\ion{Si}{II}$)$ after explosion before evoloving to have one of
% the lowest values around \tbmax. Qualitatively, this behavior is similar to
% SN\,2002bo (open diamonds in Figure~\ref{fig:r_evo}). Such evolution is
% atypical, for reference the normal SN\,2011fe is also highlighted in
% Figure~\ref{fig:r_evo} (open squares). At face value, the
% $\mathcal{R}($\ion{Si}{II}$)$ evolution suggests that the effective
% temperature of \sn\ increases significantly as it rises to maximum light. The
% optical colors (see Figure~\ref{fig:colors}), however, are nearly constant
% throughout this phase while the UV$ - $optical colors clearly decline during
% the same period, suggesting a decline in the effective temperature.
%
% \citet{Benetti04} interpret these competing effects as the result of
% significant \ion{Si}{II} mixing in the ejecta of SN\,2002bo. Mixing or
% producing Si in the outermost layers of the ejecta would (i) lead to larger
% \ion{Si}{II} velocities, (ii) produce \ion{Si}{II} line ratios that indicate
% cool temperatures (because there is less radioactive material to heat the
% ejecta), before eventually (iii) producing low values of
% $\mathcal{R}($\ion{Si}{II}$)$ as the photosphere recedes through the ejecta to
% higher temperature regions. If this interpretation is correct, this is the
% likely explanation for the line ratio evolution seen in \sn. \todo{Does TARDIS
% tell us anything about this? I have no idea...} \frommark{Not really. The TARDIS modelling verifies Si in the outer ejecta and low temperatures but more detailed modelling would be needed to test the evolution} \abi{Any chance this could
% somehow be related to burning He to Si in the shell?} \fromabi{Yes. I think it's not unreasonable to postulate that since the He shell produces IMEs that its perhaps possible a lot of that can sit in Si. Especially once you invoke that my shells are maximally burned because of our nuclear network and 1D methods so we only expect to see more IMEs as we examine a more careful handling of the nucleosynthesis and look at line of sight differences. However this would require more Si in the shell ashes than I see currently in my models, but it's defineitly a way to get in in the outer layers} \fromkate{Devil's
% advocate on this: if the idea is that the outer layers are cooler and the
% photosphere recedes then there's no reason why we wouldn't see this effect in
% all SNe Ia since all of their velocities decrease with time? }




\kate{do you want to say anything about \ion{Ca}{II}?}


% In Figure~\ref{fig:spec_comp} we compare the spectral evolution of \sn\ to
% SN\,2002bo and SN\,2010jn (both \citet{Branch06} Broad Line SNe like \sn).
% The evolution of \sn\ and SN\,2002bo is remarkably similar at all phases. The
% only significant difference between the two is the absorption trough at
% $\sim$4800\,\AA\ in pre- and maximum-light spectra. This feature, which is
% typically attributed to a combination of \ion{Fe}{II}, \ion{Fe}{III}, and
% \ion{Si}{II}, is extremely narrow in \sn. We believe this is due to a lack of
% Fe absorption as the spectra at this epoch show little evidence for iron group
% elements (IGEs). \frommark{As Kate mentioned previously, I would include discussion of TARDIS models earlier because we can also say here that there is no Fe in the outer ejecta of the model}. SN\,2010jn, which exhibits large \ion{Si}{II} velocities
% like \sn, shows weaker IME absorption and stronger IGE absorption than \sn.



% The left panel of Figure~\ref{fig:spec_comp} also shows the spectrum of
% SN\,1986G, a \citet{Branch06} Cool SN that is sometimes referred to as
% ``transitional'' given its intermediate properties between normal SNe\,Ia and
% the sub-luminous SN\,1991bg-like population (e.g., \citealt{Pastorello07}).
% The \ion{Si}{II} line ratios, and narrow $\sim$4800\,\AA\ feature in SN\,1986G
% are similar to \sn, confirming the cool nature of the photosphere at this
% early epoch. The $-12.0$\,d spectrum of \sn\ additionally shows a weaker blue
% line in the \ion{S}{II} ``W'' absorption feature at $\sim$5400\,\AA, which is
% also consistent with cool temperatures \citep{Nugent95}. \todo{86G had E(B-V)
% ~ 0.9, correct spectra for that?} \fromkate{yes.}
%
% The maximum-light spectra shown in the second panel of
% Figure~\ref{fig:spec_comp} reveal much higher temperatures for \sn, as the
% \ion{Si}{II} $\lambda$5972\,\AA\ absorption has nearly disappeared by this
% epoch. The appearance of \sn, SN\,2002bo, and SN\,2010jn (iPTF\,10ygu) are all
% similar at this epoch, with the exception of the 4800\,\AA\ feature mentioned
% above. We also show a maximum light spectrum of the \citet{Branch06} Cool,
% transitional SN\,2004eo \citep{Pastorello07}. SN\,2004eo has a similar
% appearance to \sn, though the lower velocities and cooler temperatures (as
% traced by both \ion{Si}{II} $\lambda$5972\,\AA\ and \ion{S}{II}) more clearly
% reveal the IGE absorption around 4800\,\AA. \todo{anythingto say about
% differences at $\sim$4000\,\AA?}
%
% The third panel of Figure~\ref{fig:spec_comp} shows spectra obtained roughly a
% week after maximum light. By this phase absorption due to IGE in \sn\ is
% clear. Additional differences between \sn\ and SN\,2002bo can be seen at this
% phase. There is stronger absorption in \sn\ blueward of \ion{Ca}{II} H\&K, and
% the \ion{S}{II} ``W'' absorption feature is still present in \sn\ and it
% cannot be identified in SN\,2002bo or SN\,2010jn. SN\,2004eo maintains an
% appearance that is somewhat similar to \sn, though as before, the temperatures
% are cooler and the velocities lower.
%
% Spectra obtained $\sim$6 weeks after maximum light are shown in the fourth
% panel of Figure~\ref{fig:spec_comp}. At this phase, as the SNe are
% transitioning into a nebular appearance, the appearance of each spectrum is
% similar modulo some minor differences in the relative line strengths of
% different features. \todo{add more discussion here? - I don't know a ton about
% this phase of evolution}



\section{A Physical Explanation for \sn}\label{sec:models}

The most striking feature of \sn\ is the observed UV/optical peak that occurs
shortly after discovery (Figure~\ref{fig:p48}). Any model to explain \sn\ must
account for this highly unusual feature. A UV decline in the early phase of a
SN\,Ia has previously only been observed in a single event, iPTF\,14atg
\citep{Cao15}. Resolved ``bumps'' in the early optical emission of SNe\,Ia are
also rare, having only been seen in a few events: SN\,2017cbv
\citep{Hosseinzadeh17} and SN\,2018oh \citep{Shappee19,Dimitriadis19}.
\todo{possibly other cases here, but not ``clearly resolved'' in the way these
two are}

\frommb{You can check Andy Howell's review talk on early bumps here: \url{http://bps.ynao.cas.cn/xzzx/201908/t20190820_510006.html}}

\sn\ features other properties, in addition to an intial peak $\sim$17\,d
prior to \tbmax, that separate it from normal SNe\,Ia. A good model should be
able to explain the following characteristics of \sn:
%
\begin{enumerate}
    \item The early UV/optical ``flash'' (Figure~\ref{fig:p48}).
    \item The low luminosity at maximum light (\S\ref{sec:phot}). 
    \item The rapid optical decline following \tbmax\ (\S\ref{sec:phot}). 
    \item The red optical colors at all epochs (Figure~\ref{fig:colors}). 
    \item The lack of IGE in the early spectra (Figure~\ref{fig:spec_comp}).
    \item The transition from large \RSiII\ values to low values (Figure~\ref{fig:r_evo}).
    \item The large photospheric velocities (Figure~\ref{fig:vel_evo}).
\end{enumerate}
%
\fromabi{You mention a couple times in the paper that we also note a lack of C absorption. Is that a bullet point worth mentioning here?}
As noted in \S\ref{sec:phot}, the photometric evolution of \sn\ is similar to
subluminous 91bg-like SNe, however, the spectra are missing the hallmark
\ion{Ti}{II} absorption trough associated with 91bg-like SNe. Furthermore,
around \tbmax, \RSiII\ is small in \sn, whereas 91bg-like SNe feature large
values of \RSiII\ (e.g., \citealt{Branch09}). Similarly, while the spectral
appearance and evolution of \sn\ is similar to SN\,2002bo, and other
\citeauthor{Branch06} Broad Line SNe, the photometric properties are entirely
different. SN\,2002bo features a relatively slow decline [$\Delta{m}_{15}(B) =
1.13$\,mag] with a clear secondary maximum in the $I$ band \citep{Benetti04},
which is wholey different from what is observed in \sn.

If we otherwise ignore the early flash, the remaining features (2--6) in the
list above can be understood if the explosion that gave rise to \sn\ produced
a relatively small amount of \radni\ that is strongly confined to the inner
regions of the SN ejeta. A low \radni\ yield could explain the underluminous
light curve and red colors, while a highly stratified ejecta structure could
explain the lack of IGE in the early spectra as the IGE would not have been
mixed to these outter layers. Furthermore, with a highly stratified ejecta
composition, the photosphere would transition somewhat rapidly from
\radni-poor to \radni-rich, resulting in a significant change in the
luminosity/temperature of the ejecta along the lines of what we see in the
evolution of \RSiII. \todo{did not explain photospheric velocities}\fromabi{More than did not explain. The fast velocities is in conflict with the standard intuition for a low Ni ejecta.}

\citet{Magee20} developed a suite of models featuring different \radni\
structures within the SN ejecta that they then compared to early observations
of SNe\,Ia to see which models replicate what is observed in nature.
Generally, it is found in \citet{Magee20} that highly stratified models do not
match the early evolution of normal SNe\,Ia. However, when we model the early
evolution of \sn\ using the models of \citet{Magee20}, we find that the
observations are best matched by highly stratified models, as shown in
Figure~\ref{fig:Ni_mixing}. For this modelling we have excluded the first two
epochs of ZTF observations, as we consider the mechanism that produces the
early UV flash to be different from the standard \radni\ decay that powers
most SNe\,Ia. That the (normal) rising portion of the \sn\ light curve is best
matched by stratified models strengthens the support for this interpretation.

\frommark{Add caveat that we also showed in Magee+20 that not fitting the full light curve can affect the conclusions inferred, however we don't know exactly what epochs (if any) should be excluded}

\begin{figure}
    \centering
    \includegraphics[width=3.35in]{./figures/preliminary_Ni_mixing.png}
    %
    \caption{PRELIMINARY PRELIMINARY PRELIMINARY}
    %
    \label{fig:Ni_mixing}
\end{figure}

On their own, a low-\radni\ yield and highly stratified ejecta fail to explain
the UV/optical flash seen in \sn \frommb{and blue colors too?}. A large number of scenarios have been
proposed to explain early ``bumps'' or ``flashes'' in SNe\,Ia light curves,
including: interaction between the SN ejecta and the exploding WD binary
companion \citep{Kasen10a}, interaction between the SN ejecta and
circumstellar material (e.g., \citealt{Dessart14,Piro16}), shock cooling
following the shock breakout from the surface of the WD (e.g.,
\citealt{Piro10,Rabinak11}), double detonation explosions (e.g.,
\citealt{Noebauer17,Polin19}), and extended clumps of \radni\ in the SN ejecta
(e.g., \citealt{Shappee19,Dimitriadis19}). We discuss these models and their
ability to replicate observations of \sn\ below.\footnote{We do not discuss
shock breakout models as our initial detection of \sn\ occurred at $M_g
\approx -16.3$\,mag. A progenitor radius of $\sim$10$\,R_\odot$ is needed to
explain such a high luminosity \citep{Piro10,Rabinak11}, which we consider
implausible for a WD.}

\subsection{Companion Interaction}

For SD progenitors of SNe\,Ia, the SN ejecta will shock on the surface of the
non-degenerate companion giving rise to a short-lived transient in the days
after explosion. \citet{Kasen10a} provided models for the appearance of this
interaction, which is primarily dependent upon the binary separation of the
system (assuming Roche lobe overflow for the non-degenerate companion). The
observed emission following the ejecta-companion collision is dependent upon
the orientation of the system at the time of explosion relative to the line of
sight \citep{Kasen10a}.

An analytic formulation for the luminosity and effective temperature of the
emission from the companion shock is presented in \citet{Kasen10a} (see their
Equations~22 and 25). \citet{Brown12} provide an analytic function to
approximate the fractional decrease in the observed flux as a function of the
orientation of the system. We combine equations from \citet{Kasen10a} and
\citet{Brown12} to model the early emission from \sn\ as ejecta-companion
collision. We assume the interaction emits as a blackbody, and that the
electon scattering opacity $\kappa_e = 0.2$\,cm$^{2}$\,g$^{-1}$ (as in
\citealt{Kasen10a}). Assuming $z_\mathrm{SN} = 0.0094$, $E(B-V)_\mathrm{MW} =
0.018$\,mag, and $E(B-V)_\mathrm{host} = 0.032$\,mag, we compare observed flux
measurements with those predicted by the \citet{Kasen10a} model in epochs with
MJD$\,< 58849.2$ (i.e., the first $\sim$2.5\,d after discovery when emission
from the companion interaction is significantly brighter than the luminosity
due to radioactive decay).\footnote{Given that \sn\ is an unusual SN, we make
no assumptions about the SN emission. The companion-interaction model should
therefore \textit{underestimate} the observed flux as there will be a growing
contribution due to radioactive decay with time.} The model parameters,
including: the companion separation, $a$, the mass of the ejecta,
$M_\mathrm{ej}$, the velocity of the ejecta, $v_\mathrm{ej}$, the angle
between the observer, the SN, and the companion, $\theta$, and the time of
explosion, $t_\mathrm{exp}$ are constrained via a Gaussian likelihood and flat
priors (see Table~\ref{tab:priors}) using an affine-invariant
\citep{Goodman10} Markov Chain Monte Carlo (MCMC) ensemble sampler
\citep{Foreman-Mackey13}.

\begin{figure}
    \centering
    \includegraphics[width=3.35in]{./figures/sn_companion_models.pdf}
    %
    \caption{SN ejecta-companion interaction models compared with the
    UV/optical observations of \sn. Observation symbols are the same as
    Figure~\ref{fig:p48} (solid magenta squares show \textit{Swift} $uvw2$
    observations that are not shown in Figure~\ref{fig:p48}). Solid lines show
    companion interaction model predictions in each filter (the lines have the
    same colors as the corresponding symbols for each passband). The maximum a
    posteriori model is shown via the single bold lines, while other random
    draws from the posterior are shown as thin transparent lines. The shaded
    area shows observations that are excluded from the model fit. The
    overprediction of the optical flux $\sim$13.7\,d prior to \tbmax\ suggests
    that companion interaction does not explain the early flash in \sn\ (see
    text).}
    %
    \label{fig:companion}
\end{figure}

The results of this procedure are shown in Figure~\ref{fig:companion}, where
it is clear that the model presented in \citet{Kasen10a} does an adequate job
of explaining the early UV/optical emission from \sn. We find marginalized
posterior values of $a = 9.1 \pm 0.7 \times 10^{11}$\,cm, $M_\mathrm{ej} = 1.0
\pm 0.3\,M_\odot$, $v_\mathrm{ej} = 2.2 \pm^{0.5}_{0.3} \times 10^{4}$\,\kms,
$\theta = 34 \pm^{28}_{24} \deg$, and $t_\mathrm{exp}(\mathrm{MJD}) = 58845.83
\pm 0.05$ (all uncertainties are 68\% credible regions). Examination of a
corner plot of the posterior samples shows that $M_\mathrm{ej}$ is largely
unconstrained, while $v_\mathrm{ej}$ is degenerate with $\theta$ and $a$ is
degenerate with $t_\mathrm{exp}$. 

While the interaction models roughly approximate the SN emission in the
$\sim$3\,d after explosion, they significantly \textit{overestimate} the flux
immediately after the fitting window as shown in Figure~\ref{fig:companion}.
This problem is exacerbated by the fact that the models do not include
emission associated with radioactive decay, meaning the true discrepancy
between what is predicted and what is observed is even larger than what is
shown in Figure~\ref{fig:companion}. If we extend the fitting window to
include the optical observations obtained $\sim$13.75\,d before \tbmax, the
interaction models still overpredict the optical flux at this epoch. This
overprediction of the optical flux poses a challenge for the companion
interaction scenario; an inability to simultaneously match both UV and optical
observations has been noted for other SNe\,Ia with early bumps or linear rises
\citep{Hosseinzadeh17,Miller18}.

Another challenge for the companion-interaction model is the relatively faint
maximum brightness of \sn. The peak luminosity of a SN\,Ia is directly related
to the total mass of \radni\ synthesized in the explosion \citep{Arnett82},
which is, in turn, correlated with the total ejected mass (see
\citealt{Stritzinger06,Scalzo14,Scalzo14a} and references therein). As an
underluminous explosion, \sn\ likely has a relatively low (i.e.,
sub-Chandrasekhar) $M_\mathrm{ej}$. The companion-interaction model features
systems in Roche-lobe overflow; for H-rich companions thermonuclear runaway
would only be expected as the mass of the WD approaches $M_\mathrm{Ch}$
\todo{REF???}. This is clearly at odds with \sn. Sub-Chandrasekhar explosions
are possible in the double detonation scenario, whereby the WD accretes He
from the companion star. Double detonation explosions can naturally explain
the early UV/optical flash seen in \sn\ (see below), and thus there is no need
to invoke companion interaction in such a scenario. \todo{WHAT DOES EVERYONE
THINK ABOUT THE ARGUMENT IN THIS PARAGRAPH?}
\frommb{You might want to ping Suhail here (not sure how strong the Mej - Mni is given assumptions on estimating Mej from e.g. transparency timescales). From modelling point of view, details of the explosion can give quite a range of Mni for the same Mej, see e.g. \url{https://ui.adsabs.harvard.edu/abs/2013MNRAS.429.1156S/abstract} for Chandra explosions (table 3 for Mni). Whether all these models are realized in Nature is a different question though.}
% inferred separation of the companion, $a \approx 9 \times 10^{11}$\,cm.
% Assuming the WD is accreting material via Roche Lobe overflow, then the
% companion radius is $\sim$0.35$a \approx 3.1 \times 10^{11}$\,cm $\approx
% 4.5\,R_\odot$.

For the reasons above, we do not favor the ejecta-companion interaction
interpretation for \sn. \citet{Kasen10a} notes several assumptions and
approximations in the derivation of the equations used to estimate the
emission from the companion shock. It is possible that the inclusion of more
detailed physics, or additional complexity in the analytic formulation of the
models,\footnote{For example, \citet{Kasen10a} points out that the derived
equation for the luminosity of the shock interaction does not account for the
advected luminosity that would be seen in the observer frame.} could better
reconcile companion interaction models with \sn. Such improvements are beyond
the scope of this paper, leading us to explore other explanations for the
early flash.

\subsection{Ni Clumps in the SN Ejecta}

SN\,2018oh was observed with an exquisite 30\,min cadence by the
\textit{Kepler} spacecraft and showed a clearly delineated linear rise in
flux followed by a ``standard'' $t^2$ power-law $\sim$4\,d after \tfl. Models
with extended clumps of \radni\ just below the WD surface have been proposed
as a possible explanation for the initial linear rise in SN\,2018oh
\citep{Shappee19,Dimitriadis19}. The models considered in \citet{Shappee19}
and \citet{Dimitriadis19}, which build on the work of \citet{Piro16}, only
cover the first $\sim$10\,d after explosion and assume relatively simple grey
opacities. To further investigate this possibility, Magee \& Maguire (2020)
recently performed more detailed radiative transfer calculations for SNe\,Ia
with extended clumps of \radni. They then compared these models to SN\,2018oh
and SN\,2017cbv, another event with a clearly resolved bump in the early
light curve \citep{Hosseinzadeh17}.

For \sn\ we follow the procedure in Magee \& Maguire (2020) to model the
early flash and rise of the SN. Briefly, we exclude the first two epochs of
optical detections in ZTF, and identify the best-fit model to the later
evolution of the SN from the grid of models created in \citet{Magee20}.
Following the generation of this ``baseline'' model, we add clumps of \radni\
to the outer layers of the SN ejecta, and perform full radiative transfer
calculations using TURDLS+TARDIS \todo{ref}. We find that a model with
\magee{please add the clump parameters} best matches the optical observations
of \sn, as shown in Figure~\ref{fig:Ni_bullet}. While a clump of \radni\ can
broadly replicate the observed flash in the optical, like Magee \& Maguire
(2020), we find that the extended clump of \radni\ dramatically alters the
appearance of the SN at maximum light in a way that is incompatible with
observed spectra. We therefore conlcude that Ni clumps cannot explain the
early flash seen in \sn. \frommb{If clump has effects to max light observables, is the selection of a baseline model with no clump justified?}

\begin{figure}
    \centering
    \includegraphics[width=3.35in]{./figures/colour.pdf}
    %
    \caption{PRELIMINARY PRELIMINARY PRELIMINARY}
    %
    \label{fig:Ni_bullet}
\end{figure}

\subsection{Double Detonation Models}

WDs that accrete a thin shell of He can explode via a ``double detonation''
whereby explosive burning in the He shell drives a shock into the C/O core of
the WD igniting explosive C burning, which leads to a detonation that
disrupts the entire WD (e.g., \citealt{Nomoto82,Nomoto82a,Woosley94}). Such
explosions are even possible in C/O WDs that are well below the Chandrasekhar
mass (see \citealt{Fink07, Fink10} and references therein).

Recent models of double detonation explosions presented in \citet{Polin19}
show that such explosions can replicate several of the peculiar properties of
\sn, including: the early UV/optical flash, a blue to red to blue color
transition, the low optical luminosity, red colors at maximum, a lack of
unburned C in the spectra, and a lack of IGE in the early spectra \abi{Is
this last point actually true?}. However, there is no single model that
replicates each of these individual features.

The appearance of double detonation SNe\,Ia is broadly determined by two
properties: the mass of the C/O core of the WD and the mass of the He shell.
High mass WDs ($\gtrsim 1.1\,M_\odot$) produce large ($\gtrsim$14,000\,\kms)
photospheric velocities, while low mass WDs ($\lesssim 0.9\,M_\odot$) produce
less \radni\ and therefore are underluminous, relative to normal SNe\,Ia, at
peak \citep{Polin19}. That we see both of these features in \sn\ presents a
challenge for the \citet{Polin19} double detonation models. Furthermore,
thick He shells ($M_\mathrm{He} \gtrsim 0.05\,M_\odot$) produce more
pronounced UV/optical flashes shortly after explosion, particularly in
conjunction with lower mass WDs, while thin He shells ($M_\mathrm{He}
\lesssim 0.02\,M_\odot$) produce a more extreme color inversion in the days
after explosion.

We have attempted to model the evolution of \sn\ as a double detonation
explosion, following the procedure in \citet{Polin19}. We have specifically
focused on matching the photometric evolution (as noted above no models
create high-velocity ejecta and underluminous optical peaks), with particular
attention to the colors during the early flash and at maximum light. We find
that a model with $M_\mathrm{WD} = 0.94\,M_\odot$ C/O core and a
$M_\mathrm{He} = 0.04\,M_\odot$ He shell best match \sn, as shown in
Figure~\ref{fig:double_det}.

\begin{figure}
    \centering
    \includegraphics[width=3.35in]{./figures/double_det.pdf}
    %
    \caption{Comparison of \sn\ to a dobule detonation model with a C/O core
    mass $M_\mathrm{WD} = 0.94\,M_\odot$ and He shell mass $M_\mathrm{He} =
    0.02\,M_\odot$. Symbols are the same as Figure~\ref{fig:p48}. The double
    detonation model provides a good match to the \rztf\ evolution, though
    the flux in the \gztf\ and \iztf\ bands is under- and over-predicted,
    respectively.}
    %
    \label{fig:double_det}
\end{figure}

While this model adequately matches the evolution of \sn\ in the \rztf\
filter, the predictions in the \gztf\ and \iztf\ bands do not match what is
observed. The double detonation model also severely underpredicts the flux in
the UV. Synthesized spectra from the double detonation models also do not
match what is observed in \sn. The model spectra are dominated by
\ion{Si}{II} absorption, and show high-velocity absorption due to \ion{O}{I}
and \ion{Ca}{II}, similar to \sn. For this model, however, the \ion{Si}{II}
velocities are too slow, the \ion{Si}{II} $\lambda$5972 absorption is too
strong, the \ion{S}{II} absorption too weak, and there is strong \ion{Ti}{II}
absorption trough blueward of $\sim$4400\,\AA\ (see e.g., Figure~8 in
\citealt{Polin19}). Nuclear burning in the He shell creates heavy elemnts in
the outermost ejecta of double detonation explosions, leading to deep
\ion{Ti}{II} troughs and other blanketing in the blue-optical. As was the
case for models with extended clumps of \radni, the lack of such absorption
in \sn\ poses a challenge for the double detonation model.

It is possible that improvements in the double detonation models could lead
to better agreement with \sn. For instance, the nuclear reaction networks in
the thick shells \abi{could be improved? 2D effects? please provide
additional details here. Also - any ideas about improving match in the UV?}



Alternatively, He-shell burning could produce IMEs, such as Si, that would
show up in the early spectra while the maximum light spectra would be
dominated by ejecta produced in core nucleosynthesis. \abi{Any chance this
could somehow be related to burning He to Si in the shell?} \fromabi{Yes. I
think it's not unreasonable to postulate that since the He shell produces IMEs
that its perhaps possible a lot of that can sit in Si. Especially once you
invoke that my shells are maximally burned because of our nuclear network and
1D methods so we only expect to see more IMEs as we examine a more careful
handling of the nucleosynthesis and look at line of sight differences. However
this would require more Si in the shell ashes than I see currently in my
models, but it's defineitly a way to get in in the outer layers}


\section{Rate of Thick He Shell Double Detonation Events}\label{sec:rates}

\textbf{This may not actually be a thick shell}

\aam{Use rough numbers from CLU and simple binomial calculation}

\section{Discussion and Conclusion}\label{sec:conclusions}

\todo{be clever}

\acknowledgements

A.A.M.~is funded by the Large Synoptic Survey Telescope Corporation, the
Brinson Foundation, and the Moore Foundation in support of the LSSTC Data
Science Fellowship Program; he also receives support as a CIERA Fellow by the
CIERA Postdoctoral Fellowship Program (Center for Interdisciplinary
Exploration and Research in Astrophysics, Northwestern University).

This work is based on observations obtained with the Samuel Oschin Telescope
48-inch and the 60-inch Telescope at the Palomar Observatory as part of the
Zwicky Transient Facility project. ZTF is supported by the National Science
Foundation under Grant No. AST-1440341 and a collaboration including Caltech,
IPAC, the Weizmann Institute for Science, the Oskar Klein Center at Stockholm
University, the University of Maryland, the University of Washington,
Deutsches Elektronen-Synchrotron and Humboldt University, Los Alamos National
Laboratories, the TANGO Consortium of Taiwan, the University of Wisconsin at
Milwaukee, and Lawrence Berkeley National Laboratories. Operations are
conducted by COO, IPAC, and UW.

MMT Observatory access was supported by Northwestern University and the
Center for Interdisciplinary Exploration and Research in Astrophysics (CIERA).

We acknowledge the use of public data from the \textit{Swift} data archive.

\software{
          \texttt{astropy} \citep{Astropy-Collaboration13},
          \texttt{scipy} \citep{Jones01}, 
          \texttt{matplotlib} \citep{Hunter07},
          \texttt{pandas} \citep{McKinney10},
        %   \texttt{emcee} \citep{Foreman-Mackey13},
        %   \texttt{corner} \citep{Foreman-Mackey16},
          \texttt{SALT2} \citep{Guy07},
          \texttt{sncosmo} \citep{Barbary16}
          }

%% For this sample we use BibTeX plus aasjournals.bst to generate the
%% the bibliography. The sample63.bib file was populated from ADS. To
%% get the citations to show in the compiled file do the following:
%%
%% pdflatex sample63.tex
%% bibtext sample63
%% pdflatex sample63.tex
%% pdflatex sample63.tex

\bibliography{/Users/adamamiller/Documents/tex_stuff/papers}
\bibliographystyle{aasjournal}

%% This command is needed to show the entire author+affiliation list when
%% the collaboration and author truncation commands are used.  It has to
%% go at the end of the manuscript.
%\allauthors

%% Include this line if you are using the \added, \replaced, \deleted
%% commands to see a summary list of all changes at the end of the article.
%\listofchanges

\begin{deluxetable*}{ccccc}
\tabletypesize{\scriptsize}
\tablewidth{0pt}
\tablecaption{Log of Spectroscopic Observations for SN~2019yvg.\label{tab:spectra}}
\tablehead{
\colhead{Date (UT)}&
\colhead{MJD}&
\colhead{Phase$^{*}$}&
\colhead{Telescope+Instrument}&
\colhead{Range}\\
\colhead{}&
\colhead{(days)}&
\colhead{(days)}&
\colhead{}&
\colhead{(\AA)}}
\startdata
31 Dec 2019 &  58848.273298   &    &  LT+SPRAT$^{*}$ &  4000--9000?   \\
03 Jan 2020 &     &     &  LT+SPRAT$^{*}$ &  4000--9000   \\
04 Jan 2020 &     &     &  LT+SPRAT &  4000--9000   \\
12 Jan 2020 &     &     &  LT+SPRAT &  4000--9000   \\
12 Jan 2020 &     &     &  LT+SPRAT  &  4000--9000   \\
15 Jan 2020 &     &     &  P60+SEDM$^{**}$ &  4000--9000   \\
18 Jan 2020 &     &     &  P60+SEDM &  4000--9000   \\
24 Jan 2020 &     &     &  MMT+Binospec$^{***}$ &  4000--9000   \\
25 Jan 2020 &     &     &  Keck+LRIS$^{****}$ &  4000--9000   \\
27 Jan 2020 &     &     &  P60+SEDM  &  4000--9000   \\
29 Jan 2020 &     &     &  NOT+ALFOSC$^{******}$ &  4000--9000   \\
01 Feb 2020 &     &     &  P60+SEDM &  4000--9000   \\
\enddata
\tablenotetext{*}{Liverpool Telescope}
%\tablenotetext{**}{}
%\tablenotetext{***}{}
\end{deluxetable*}



\end{document}